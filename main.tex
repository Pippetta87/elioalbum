\documentclass[oneside,12pt,fleqn]{memoir}
\usepackage{makeidx}
%\usepackage[utf8]{inputenc}
\pagestyle{plain}
%%%%%%%%%%%%%%%%%%%%%%%%%%%%%%%%%%% importa pacchetti
\usepackage{usepkg}
\usepackage{fancyfoot}
%%%%%%%%%%%%%%%%%%%%%%%%%%%%%%%%%%%%%
%%%%%%%%%%%%%%%%%% titletoc, titlesec setting
\usepackage{titleT}
%%%%%%%%%%%%%%%%%% setlength
\usepackage{mylength}
\linespread{0.5}
%%%%%%%%%%%% Hyperref package
\usepackage{hyperref}
\hypersetup{
    colorlinks,
    citecolor=black,
    filecolor=black,
    linkcolor=black,
    urlcolor=black
}
%%%%%%%%%%%%%%%%%%%%%%%%
%%%%%%%%%%%%%%%%%Geometry package
\usepackage{mygeometry}
%%%%%%%%%%%%%%%%%%%%%%%%%%%%%%%%%%% Funzioni per questo file main
\usepackage{LocalF}
%%%%%%%%%%%%%%%%%%%%%%%%%%%%%%%%%%% Funzioni generali
\usepackage{functions}
\usepackage{mathOp}
%http://tex.stackexchange.com/questions/246/when-should-i-use-input-vs-include
\usepackage{sources}
%%%%%%%%%%%%%%%%%%%%%%%%%%%%%%%%%

\makeindex
\raggedbottom %http://tex.stackexchange.com/questions/102084/annoying-paragraph-spacing-issue-with-memoir

\author{Pippetta}
\title{Abb tesi eliosismologia}
\date{\today}

\begin{document}

\frontmatter


\maketitle


\tableofcontents*
\listoffigures

\mainmatter



\part{Cosa fare?}


\subfile{intro}

\part{Tesina}



\chapter{Introduzione al modello solare.(alle oscillazioni solari.)}
\PartialToc


\section{Eliosismologia.}

\subsection{Per punti}
\begin{itemize*}
\item Onde, modi normali
\item Cos'e l'eliosismologia.
\item info about theSun from surface oscillation.
\item Model test against observations.
\item Per molti modi il limite superiore non \'e lontano dalla fotosfera: tunnel attraverso la regione evanescente nell'atmosfera visibile.
\item principali risultati:
    \item Measurement of frequencies determine value of X,Y in outer envelope.
\item Come \'e divisa la tesi.



\item Doppler Measurement intensity and oscillation in atmospheric T.

Le oscillazioni interne si manifestano come moti oscillatori in atmosfera, cambiamenti nella potenza irradiata dal sole e campiamenti delle propriet\'a delle linee spettrali causati da oscillazioni di temperatura nell'atmosfera.
\item Evoluzione strumentazione. Doppler compensator. Resonant scatteing. Spectroheliograph. Pyrheliometer.
\item Scattering cell (Fossat,ricort 75; Brookes 76; Cacciani Fofi78; Rhodes 86; Tomczyk 95)
\item Fourier tachometer (Brown 84)
\item Broadband intensity observation from space (Woodard hudson 83; toutain frohlich 92)
\item Heliosismic facilities (Duvall 95)
\item Network observing stations: Bison (81): resont scattering cell, disk averaged velocity obs., (chaplin 96); GONG (95): Fourier tachometer (harvey 96), spatially resolved velocity observations; LOWL(Tomczyk95): magnetoptical filter; IRIS (Fossat 91); TON (Chou 95).
\item Space observations. SOHO(Domingo95). GOLF: Detection low frq. modes (may be g) integrated on whole disk, resonant scattering doppler velocity sodium cell (Gabriel 95-97). VIRGO (Variability of solar irradiance and gravity oscillations): g modes in intensity data (Frohlich 95-97). SOI/MDI: Fourier tachometer, Spatial resolution of \SI{2}{\arcsec}, oscillation up to degree l=1000 (4000??)
\item  Asteroseismology and Interferometry (0709.4613)
\item Leighton62:
\item Fondere parte meccanica della sezione leggi di conservazione fino all'equazione delle perturbazioni linearizzate e periodo fondamentale.
\item Struttura di equilibrio ($\TtwoDy{t}{r}=0$). Pulaszioni: variazione equazione del moto. (Vedi collins 3.1 pg 48)


\item Quali grandezze della struttuara solare hanno effetto diretto sulle oscillazioni.

Le frequenze delle oscillazioni adiabatiche sono determinate dal profilo radiale $P,\ \rho,\ ,g,\ \Gamma_1$: assumendo valide le relazioni basilari della struttura stellare  sono indipendenti 2  quantit\'a ($\rho(r),\Gamma_1(r)$ per esempio); viceversa le frequenze forniscono informazioni derette solo su queste quantit\'a (meccaniche).

\item At observed solar frequencies the displacement at surface is approx. radial:
\begin{equation*}
\frac{\xi_h(R)}{\xi_r(R)}\approx\frac{GM}{R^3}\frac{L}{\omega^2}
\end{equation*}

\item Deubner75: power ridge in \dgndi{} with slit
\item Ridge not resolved into single mode frequency ($\Delta l=1$: Solar circumference scan length??): spherical harminics are orthogonal on the full sphere. Single frequncy value can be computed by approximating power at each l by a smooth curve and finding the max of this curve $P(l,\nu)$.
\item Stellar pulsation modulate the emergent flux by their influence on the location of, and the physical conditions at the surface $\tau=\frac{2}{3}$.
\item strumenti di misura: resonance-scattering spectrometer
\item Claverie79: integrated light
\item Quality factor $Q=\frac{\nu}{\Delta\nu}$($\approx\numrange{e3}{e4}$: weak damping)
\item evoluzione doppler measurement
\item Trasformata di fourier/frequenza di Nyquist
\item Prodotto $G\msun$
\item Incertezza su G
\item incertezze parametri standard
\end{itemize*}


L'osservazione di fenomeni periodici nelle stelle permette di dedurre informazione sulla struttura.

Le stelle nella fase di MS  sono caratterizzate da numerosi modi di oscillazione di piccola ampiezza: lo studio delle osciallazioni della superficie solare e l'estrapolazione delle informazioni sulla struttura interna in essi contenuta \'e detta eliosismologia. \'E possibile calcolare numericamente le frequenze sulla base di un modello stellare e al variare di uno o pi\'u parametri del modello analizzare la corrispondenza con quelle osservate inoltre sono state sviluppate tecniche di inversione per dedurre la struttura e la dinamica interna del sole dalla misura delle frequenze.

\subsection{Equilibrio idrostatico.}

Le frequenze sono determinate principalmente dalla stratificazione (e dinamica) della regione in cui le ampiezze sono apprezzabili. 

\subsubsection{Distrubuzione della massa.}

Considero una distribuzione di massa sferica.

La massa presente in un guscio infinitesimo 
\begin{align*}
&dm=4\pi r^2\,dr-4\pi r^2\rho v\,dt&\intertext{equivalente ad all'equazione di continuit\'a}\\
&\PDy{t}{\rho}+\nabla\cdot(\rho\vec{v})=0
\end{align*}

La forza totale agente su un volume V di superficie S \'e
\begin{equation*}
\int_V\vec{F}\rho\,dV+\int_S\vec{t}(\vec{n},\vec{x},t)\,dS
\end{equation*}
dove $\hat{n}$ \'e la normale in ciascun punto di S, e $\vec{F}$ \'e una possibile body force per unit mass.

Lo stress su una superficie con normale parallela ad un asse coordinato ($\hat{n}=\hat{e}_i$ \'e
\begin{equation*}
\vec{\sigma}_i=\vec{t}(\hat{e}_i,\vec{x},t)=(\sigma_{i1},\sigma_{i2},\sigma_{i3}
\end{equation*}

quindi si ha che gli elementi $\sigma_{ij}(\vec{x},t)$ formano un tensore.

La conservazione del momento richiede
\begin{align*}
&\rho\TDy{t}{\vec{v}}=-\nabla\tensor{P}{_\cdot_\cdot}+\rho\vec{f}\\
&\intertext{per pressione termodinamica cio\'e trascurando viscosit\'a molecolare e radiativa, campi magnetici e turbolenze}\\
&\rho\TDy{t}{\vec{v}}=-\nabla P+\rho\vec{f}
\end{align*}
Ottengo quindi la condizione di equilibrio idrostatico $\ddvec{r}=0$:
\begin{equation}
\nabla P=\rho f\label{eq:idrosta}
\end{equation}


\subsubsection{Potenziale gravitazionale.}

Esplicito la forma della forza per unit mass f:
\begin{equation}\label{eq:gravitya}
g=\frac{Gm(r)}{r^2}
\end{equation}

Il potenziale gravitazionale \'e soluzione dell'equazione di Poisson 
\begin{align}
&\nabla^2\Phi=4\pi G\rho\label{eq:poisson}\\
&g=\PDy{r}{\Phi}=\frac{Gm(r)}{r^2}
\end{align}

Sostituendo nell'equazion di equilibrio idrostatico
\begin{equation}
\TDy{r}{P}=-\frac{Gm(r)\rho(r)}{r^2}\label{eq:idrostae}
\end{equation}


\subsubsection{Tempo di evoluzione dinamico.}

Scrivo l'equazione del moto per unit\'a di superficie di un guscio sferico
\begin{align*}
&\frac{dm}{4\pi r^2}\PtwoDy{t}{r}=f_P+f_g&\intertext{f \'e una forza per unit\'a di superficie,}\\
&\frac{1}{4\pi r^2}\PtwoDy{t}{r}=-\PDy{m}{P}-\frac{Gm}{4\pi r^4}
\end{align*}
Il valore tipico della derivata \'e approssimato dal rapport delle quantit\'a
\begin{align*}
&\tau_{ff}\approx\sqrt{\frac{R}{g}}\\
&\tau_{esp}\approx R\sqrt{\frac{\rho}{P}}
\end{align*}

Nelle stelle in cui l'equilibrio idrostatico \'e una buona approssimazione il tempo caratteristico di reazione a perturbazione dell'equilibrio \'e

\begin{align*}
&\tau_{idro}\approx \sqrt{\frac{R^3}{GM}}\approx\frac{1}{2}(G\overline{\rho})\expy{-\frac{1}{2}}\\
&G\msun=\num{1.32712440018e20}\pm\num{8e9}\si{\cubic\meter\per\square\second}\\
&\tau_{idro}^{\odot}\approx\SI{27}{\minute}
\end{align*}

\subsection{Teorema del viriale: equilibrio idrostatico.}

\begin{align*}
&\frac{1}{2}\TtwoDy{t}{I}=2K+\Omega\\
&2K=\sum_im_iv_i^2=\sum_i\scap{p_i}{v_i}&\intertext{$\scap{p_i}{v_i}$ measure rate of momentum transfer hence must be related to Pressure}\\
&P=\frac{1}{3}\int_pn(\vec{p})\scap{p}{v}d^3p&\intertext{confrontando le ultime due equazioni si ha:}\\
&2K=3\int_VP\,dV=3\int_M\frac{P}{\rho}\,dm(r)
\end{align*}

Il teorema del viriale si riscrive

\begin{align*}
&\frac{1}{2}\TtwoDy{t}{I}=\int_M\frac{3P}{\rho}\,dm(r)+\Omega&\intertext{Per equilibrio idrostatico}\\
&\frac{1}{2}\TtwoDy{t}{I}=0
\end{align*}

\subsection{\texorpdfstring{$\gamma$-law }{gamma-law} equation of state.}

Se vale una relazione del tipo $P=(\gamma-1)\rho u$ (per i gas ideali monoatomici con $\gamma=\frac{5}{3}$, $P=\frac{2}{3}\rho u$)
\begin{equation*}
2K=3(\gamma-1)\int u\,dm
\end{equation*}

$K=E_i$ solo per $\gamma=\frac{5}{3}$: l'energia cinetica \'e uguale all'energia interna totale solo in determinate circostanze.

Il teorema del viriale si riscrive
\begin{equation*}
3(\gamma-1)E_i+\Omega=0
\end{equation*}

e scrivendo l'energia totale $W=E_i+\Omega$ ottengo la relazione esplicita tra energia totale ed energia potenziale gravitazionale per stelle idrostatiche in cui vale la relazione $P=(\gamma-1)\rho u$
\begin{equation*}
W=\frac{3\gamma-4}{3(\gamma-1)}\Omega
\end{equation*}

\subsection{Teorema del viriale.}

Il teorema del viriale esprime una relazione statistica tra particelle interagenti: in particolare ricavo una relazione tra energia interna e energia potenziale gravitazionale.

L'energia potenziale gravitazionale della stella
\begin{equation}
\Omega=-\int_0^M\frac{Gm(r)}{r}\,dm\label{eq:energiapg}
\end{equation}

L'energia interna per unit\'a di massa per un gas ideale (monoatomico) u si esprime
\begin{equation}
\frac{P}{\rho}=\frac{R}{\mu}T=(\gamma-1)c_vT=\frac{2}{3}u\label{eq:energiaigp}
\end{equation}
Il teorema del viriale stabilisce che, posto $E_i=\int_0^Mu\,dm$,
\begin{equation}
E_g=-2E_i\label{eq:virialegpm}
\end{equation}

Per un'equazione di stato generale definisco il parametro $\zeta$
\begin{equation}
\zeta u=3\frac{P}{\rho}
\end{equation}
Per un gas ideale $\zeta=3(\gamma-1)\xrightarrow{\gamma=\frac{5}{3}}2$.

Per un gas di fotoni
\begin{align}
&P=\frac{1}{3}aT^4\label{eq:pressurephg}\\
&u\rho=aT^4,\ \zeta=1
\end{align}

Per $\zeta$ costante nella stella il teorema del viriale prende la forma

\begin{equation}
\zeta E_i+E_g=0\label{eq:virialezetac}
\end{equation}


\subsection{Particelle interagenti tramite potenziale funzione omogenea di grado n delle coordinate}

\begin{align*}
&2K-nU-3PV=0\\
&E=U+K&\intu{energia totale (energia interna?)}\\
&(n+2)K=nE+3PV
\end{align*}


\subsection{Modo fondamentale di oscillazione}

Le oscillazioni solari sono in prevalenza acustiche, legate al gradiente della pressione, e quindi determinate dal profilo radiale della velocit\'a del suono.


Per un corpo in equilibrio idrostatico ricavo il valore medio della velocit\'a del suono utilizzando il teorema del viriale
\begin{align*}
    &-\Omega=3\int_VP\,dV=3\int_M\frac{P}{\rho}\,dm=3\int_M\frac{v_s^2}{\Gamma_1}\,dm\\
    &=3\exv{\frac{v_s^2}{\Gamma_1}}M\approx3\frac{\overline{v}_s^2}{\gamma_1}M
\end{align*}

Se scrivo $\Omega=q\frac{GM^2}{R}$, per stelle di sequenza principale ho che $q\approx1.5$.

e quindi per il modo fondamentale di oscillazione radiale
\begin{align*}
    &\lambda_1\approx 2\rsun{}\\
    &\omega_1\approx\frac{c}{\lambda_1}\approx\SI{1}{\hour}
\end{align*}

\begin{todo}{considerazioni su $\Gamma_1$}
periodo fondamentale tenendo conto di $\Gamma_1$ 
\end{todo}

\subsection{Oscillazioni dei 5 minuti.}

Leighton62 osserva che la superficie solare ha scale spazio-temporali privilegiate: in particolare \'e presente un comportamento periodico nell'atmosfera a tutte le altezze rilevato tramite effetto doppler. Il periodo \'e di circa 300 secondi.

Il modello proposto da Ulrich70 e stein leibacher 71 considera le propriet\'a delle perturbazioni all'interno del Sole, in particolare dalla relazione di dispersione per onde acustiche si ha la definizione di cavit\'a risonanti al di sotto della superficie solare: sono possibili onde stazionarie per determinati valori di  $(k_h,\omega)$, dove $k_h$ \'e il numero d'onda orizzontale.

\subsection{Modi di oscillazione (Onde Stazionarie). Cavit\'a risonanti.}

I modi osservati hanno $\nu\geq\SI{500}{\micro\hertz}$: sono modi p (onde stazionarie: oscillazioni velocit\'a temperatura sfasati di \ang{90}) e modi f di alto grado (onde di gravita di una superficie libera).

Le vibrazioni libere di un corpo finito o comunque con condizioni ai bordi sono onde stazionarie la cui parte reale \'e del tipo $f(x,y,z)\cos{(\omega t+\alpha)}$: in assenza di effetti dissipativi la velocit\'a di fase \'e nulla e la velocit\'a di gruppo infinita.


Un'onda stazionaria in direzione radiale implica che  l'integrale di $k_r$ nella regione di propagazione sia un intero multiplo di $\pi$

\begin{align}
&(n+\alpha)\pi\approx\int_{r_t}^Rk_r\,dr\approx\int_{r_t}^R\frac{\omega}{c}\sqrt{1-\frac{S_l^2}{\omega^2}}\,dr&\intertext{ho usato la relazione di dispersione per onde acustiche e la frequenza di Lamb $S_l$}\\
&\omega^2=c^2|\vec{k}|^2,\ S_l^2=\frac{l(l+1)c^2}{r^2}
\end{align}

quindi il perido di un modo con $k_h$ fissato \'e determinato da

\begin{align}
    &(n+\frac{1}{2})\pi=\int k_r\,dr=\omega\int\frac{dr}{c}=\frac{2\omega^2}{(\gamma-1)gk_h}&\intertext{quindi si ha una curva parabolica compatibile con quelle osservate:}\\
    &\omega_n^2=\frac{(n+\frac{1}{2})\pi(\gamma-1)gk_h}{2}\approx(n+\frac{1}{2})gk_h&\intertext{dove l \'e il numero di lunghezze d'onda in una circonferenza solare:}\\
    &\lambda_h=\frac{2\pi}{k_h}\approx\frac{2\pi R}{\sqrt{l(l+1)}}
\end{align}


\begin{usefull}{Stima profondit\'a cavit\'a acustica}

La profondit\'a della cavit\'a acustica varia con il variare della scala orizzontale dell'onda: considero una stratificazione adiabatica

\begin{align*}
    &T=\Dcvar{\TDy{z}{T}}{Ad}\delta&\intertext{$\delta$ \'e la profondit\'a sotto la fotosfera}\\
    &\Dcvar{\TDy{z}{T}}{Ad}=\frac{T}{P}\TDly{P}{T}|_{Ad}\TDy{z}{P}=\frac{\Gamma_2-1}{\Gamma_2}\frac{\mu}{R}g=\frac{g}{c_P}&\intertext{$c_P$ \'e il calore specifico a pressione costante per unit\'a di massa. Scrivendo $c^2=(\Gamma_3-1)g\delta$, da $c=\frac{\omega}{k_h}$ al raggio per cui $k_r=0$ segue:}\\
    &\delta=\frac{\omega^2}{k_h^2(\Gamma_3-1)g}
\end{align*}

I modi con stesso $\frac{\omega}{k_h}$ sono confinati nella stessa cavit\'a.

\end{usefull}

L'analisi tramite FFT (della frequnza e del numero d'onda) delle osservazioni della superficie solare di deubner75 confermano che la  potenza delle oscillazioni (con k piccolo: $k=\frac{2\pi}{\lambda}<\SI{1}{\per\mega\meter}$) si distribuisce in linee determinate nel diagramma $(k_h,\omega)$ predette dal modello, mostra che sono provocate da modi acustici non radiali degli strati interni alla fotosfera: la concentrazione della potenza a bassi numeri d'onda indica che siamo in presenza di un fenomeno globale.

In particolare vengono effettuate delle scansioni lineari per \SI{300}{\arcsec} sulla superficie solare ogni \SI{110}{\second} con un'apertura di $2.0\times2.5$ \si{\arcsec}, e tramite lo shift Doppler della line del CI $5380$ viene misurata la velocit\'a lungo la linea di vista.

La larghezza dello spettro risonante ($Q=\frac{\Delta\nu}{\nu}$ \'e il rapporto tempo di crescita(o tempo di smorzamento)/periodo) riflette la legge di dispersione e la rapidit\'a di crescita/dissipazione di alcuni modi nella bassa fotosfera piuttosto che casualit\'a del processo.



Claverie 1979/80 osserva nello spettro Doppler (Neutral K line: \SI{769.9}{\nano\meter}) della luce integrata sull'intero disco solare delle frequenze equispaziate circa \SI{68}{\micro\hertz} interpretate come modi p di alto ordine l e basso grado l.

\begin{todo}{A resonant scattering spectrometer}

\end{todo}

\begin{todo}{Integrated sunlight}
grec83 moltitude and sharpness of the line in the solar low-l oscillation spectrum
\end{todo}

\subsection{Analisi modale.}

Osservando il campo di velocit\'a $v(x,y,t)$ sulla superficie solare ottengo la distribuzione della potenza delle oscillazioni $P(k_x,k_y,\omega)$

\begin{align}
    &v(x,y,t)=\int f(k_x,k_y,\omega)\exp{i(k_xx+k_yy+k_zz+\omega t)}\,dk_x\,dk_y\,d\omega\\
    &P(k_x,k_y,\omega)=ff^*\\
    &P(k_h,\omega)=\frac{1}{2\pi}\int_0^{2\pi}P(k_h\cos{\phi},k_h\sin{\phi},\omega)\,d\phi&\intertext{non esiste direzione privilegiata sulla superficie: la dipendenza \'e solo da $k_h=\sqrt{k_x^2+k_y^2}$.}
\end{align}

Un segnale di durata T permette una risoluzione $\Delta\omega=\frac{2\pi}{T}$: se devo risolvere due frequenze $\omega$ e $\omega+\Delta\omega$ devo osservare per un tempo $T=\frac{2\pi}{\Delta\omega}$ la frequenza pi\'u bassa osservabile \'e $\Delta\omega$, il limite superiore delle frequenze osservate \'e dato dalla risoluzione temporale $\Delta t$, la frequenza di Nyquist $\omega_{Ny}=\frac{\pi}{\Delta t}$ e analogamente per le variabili spaziali e numero d'onda associato
\begin{align}
&\Delta\omega=\frac{2\pi}{T}\leq\omega\leq\frac{\pi}{\Delta t}\\
&\Delta k_x=\frac{2\pi}{L_x}\leq k_x\leq\frac{\pi}{\Delta x}
\end{align}


Quando la dimensione dell'area osservata \'e comparabile con il disco solare tengo conto della geometria sferica
\begin{align}
    &v(\theta,\phi,t)=\sum_{l=0}^{\infty}\sum_{m=-l}^la_{lm}(t)Y_{lm}(\theta,\phi)\\
    &P(l,\nu=\frac{\omega}{2\pi})=a(\omega)a(\omega)*&\intertext{$a(\omega)$ \'e la trasformata di Fourier di $a_{l0}(t)$.}
\end{align}

Pi\'u precisamente il segnale \'e proporzionale alla velocit\'a proiettata lungo la linea di vista. Per modi con l basso o intermedio le oscillazione sono in direzione radiale. Prendendo l'asse delle armoniche sferiche sul piano del cielo ortogonale alla linea di vista, il segnale Doppler osservato \'e

\begin{equation}
    V_D(\theta,\phi,t)=\sin{\theta}\cos{\phi}\sum_{n,l,m}A_{nlm}c_{lm}P_l^m(\cos{\theta})\cos{(m\phi-\omega_{nlm}t-\beta_{nlm})}
\end{equation}
il fattore $\sin{\theta}\cos{\phi}$ deriva dalla proiezione della velocit\'a radiale sulla linea di vista.

Per isolare il contributo di una singola $Y_{l_0m_0}$ considero
\begin{align}
    &V_{l_0m_0}(t)=\int_AV_D(\theta,\phi,t)W_{l_0m_0}(\theta,\phi)\,dA\\
    &=\sum_{n,l,m}S_{l_0m_0,lm}A_{nlm}\cos{(\omega_{nlm}t+\beta_{nlm,L_0m_0})}\\
    &S_{l_0m_0,lm}\propto\delta_{ll_0}\delta_{mm_0}&\intu{funzione di risposta,}\\
    &W_{l_0m_0}\approx Y_{l_0m_0}
\end{align}

In pratica $V_{l_0m_0}(t)$ contiena contributi da valori di $(l,m)$ vicini.

La trasformata di Fourier di $V_{l_0m_0}(t)$ permette di isolare i signoli modi caratterizzati dall'ordine radiale n.


In linea di principio:
\begin{itemize}
    \item Dall'andamento di un modo sulla superficie solare si ricava $(l,m)$.
    \item L'ordine radiale n si ricava dalla distribuzione delle frequenze di oscillazione.
\end{itemize}


\subsection[???]{Risoluzione delle osservazioni.}


Large telescopes, dedicated to long seismic observations. Stable sensitive detector.

Doppler velocities/intensity fluctuations: p modes, Stable doppler image of entire disk; g modes, separate from noise of earth atmosphere and solar convective motion.

Due modi separati di $\Delta\nu$ sono risolti con osservazione di $T\geq\frac{1.5}{\Delta\nu}$ (T(hour)=417/($\Delta\nu(\si{\micro\hertz})$))

\begin{itemize}
    \item even/odd l: $T\geq \SI{6}{\hour}$
    \item modi l: $T\geq \SI{40}{\hour}$
    \item Rotational splitting: $T\geq \SI{400}{\hour}$
\end{itemize}

L'atmosfera solare \'e sudivisa in photosfera circa \SI{100}{\kilo\meter} e cromosfera, la parte pi\'u esterna: nella fotosfera il gas cambia da quasi trasparente a completamente opaco. La luce che riceviampo dal Sole \'e emessa dalla fotosfera.

From space helio (Toutain):

\begin{itemize}
    \item Ground network: Bison, Iris, Gong, Ton.
    \item Soho (Space)
\end{itemize}

Fino anni '80:

\begin{itemize}
    \item Ground: interruption N/D, Whether cond.
    \item first round clock observation grec81.
    \item filling method grec80.
    \item Osservazioni ininterrotte aumentano la risoluzione in $\omega$ e il rapporto S/N per modi con tempo di vita maggiore del tempo di osservazione.
    \item Rotational splitting $\Delta\nu=0.45\si{\micro\hertz}$.
    \item lines of modes below \SI{2}{\milli\hertz} have lifetime approx 1 month.
    \item Il rumore solare aumenta con la frequenza.
    \item Effetto dell'atmosfera: osservazioni di basso l hanno rumore a basse frequenze.
    \item Alto l sono affette da perdita di coerenza sul disco solare: leakage of high degree modes ($l>300$ sono coperte gi\'a da seeing di \ang{;;4}).
\end{itemize}

--(Space)--

ACRIM (active cavity radiometer irradiance monitor).

\begin{itemize}
    \item Orbital period \SI{95}{\minute} (\SI{35}{\minute} notte ??): Side band at \SI{+-170}{\micro\hertz}.
    \item Total irradiance measure: accuratezza maggiore di $0.1\%$ (possibile identificare i modi p in luce integrata).
    \item Osservazioni di 10 mesi: spettro modi p (woodard84) $l=0,1,2$, in un range di frequenze $\nu=\numrange{2.5}{3.8}\si{\milli\hertz}$.
    \item shutter cycle 131 secondi: occuratenza nelle frequenze di \SI{0.4}{\micro\hertz}.
    \item S/N da 1-4.
\end{itemize}

IPHIR (PhobosII).

\begin{itemize}
    \item Misure di luce integrata a 3 lunghezze d'onda: 3 interference filter \SIlist{335;500;862}{\nano\meter}.
    \item Accuratezza circa 1ppm.
    \item S/N circa 20 per modi p a \SI{3}{\milli\hertz}.
    \item p-modes ($l=0,1,2$): $\nu=\SIrange{2.4}{3.8}{\milli\hertz}$.
    \item Amplitude changes strongly with time (p-mode are stochastically excited by turbolent convection).
\end{itemize}

Virgo(SOHO: L1 Sole-Terra).

\begin{itemize}
    \item modi p di basso grado 
    \item Spectral and total irradiance: radiometer, fotometro con filtri a interferenza, Si-diode detector: $l\leq3$.
    \item Loi: $l\leq7$.
\end{itemize}

Golf(SOHO)
\begin{itemize}
    \item Misura spostamento Doppler di luce integrata sul disco solare: vapori di sodio (linee di Na: D1, D2).
    \item Modi p e g di basso grado angolare.
\end{itemize}

SOI/MDI(SOHO)
\begin{itemize}
    \item MDI (Michelson doppler image): fourier tachometer tuned across Ni-line (\SI{676.8}{\nano\meter}).
    \item Modi p con grado angolare medio-alto $L\leq4000$.
    \item Correlazione dei segnali velocit\'a/intensit\'a: effetti non adiabatici nella fotosfera.
\end{itemize}


\section{Onde}

\subsection{Onde EM}

Equazione di Laplace in coordinate sferiche: Jackson pg 95

Power losses in a cavity: Q. Jackson pg 371

\section[(ei fu)]{(era...)Leggi di Conservazione. Equilibrio statico.}
\begin{itemize*}
\item fusione di questa sezione?
\item Equazioni mechaniche nella sezione 1?
\item Equazioni conservazione e trasporto di energia parte 3?
\item \sout{Equazioni di base della struttura stellare}
\item Equazion of motion for spherical symmetry: $\tau_{ff}$, $\tau_{expl}$ ($\S 2.4$ kipp): la stella occupa stati di quasi equilibrio per gran parte della vita $\tau_{nucl}$
\item Kelvin-Helmholtz scale time ($\S 3.1-3.3$ kippen):
\item Equazioni par 4 cox (nella parte equilibrio struttura autogravitante): leggi di conservazione
\item Equazion of motion for spherical symmetry: $\tau_{ff}$, $\tau_{expl}$ ($\S 2.4$ kipp): la stella occupa stati di quasi equilibrio per gran parte della vita $\tau_{nucl}$
\item \sout{Equazioni struttura solare. Simmetria sferica: cosa trascuro.}
\begin{align*}
&\TDy{r}{p}=-\frac{Gm\rho}{r^2}&\intu{Momentum conservation along with Poisson's equation:}\\
&\TDy{r}{m}=4\pi r^2\rho\\
&\TDy{r}{T}=\nabla\frac{T}{p}\TDy{r}{p}\\
&\TDy{r}{L}=4\pi r^2[\rho\epsilon-\rho\TDof{t}\frac{u}{\rho}+\frac{p}{\rho}\TDy{t}{\rho}]
\end{align*}
\item Trascuro la rotazione e i compi magnetici.
\item \sout{The assumption $\ten{P}=IP$ where P is the thermodynamic pressure imply neglegible molecular and radiative viscosity, large-scale magnetic field and turbolence.}
\item \sout{temposcala dinamico:}
\item Connection between convective energy transport and local structure.
\item Convective instability: $\nabla_{Rad}>\nad{}=\Dcvar{\PDly{P}{T}}{Ad}$.
\item $\frac{1}{\kappa\rho}$ is the mean free path of photon.
\item Where energy is transported by radiation: $\nabla=\nabla_{Rad}=\frac{3}{16\pi acG}\frac{\kappa P}{T^4}\frac{l(r)}{m(r)}$.
\item onde propagazione frequenze plasma lunghezze caratteristiche frequenze di taglio (asymptotic description)
\item Plasma ideale: costante di accoppiamento
\item equazione di stato (stix pg 29) ???
\item The applicability of fluid approach (sh8u gas dynamics). mean free path and plasma frequency (sh8u gas dynamics)
\item Helium diffusion in the Sun
\begin{equation*}
    \PDy{t}{X}=R_H+\frac{1}{r^2\rho}\PDof{r}[r^2\rho(D_H\PDy{r}{X}+V_HX)]
\end{equation*}
$R_H$ Rate of change of H abbundance due to nuclear reactions, $D_H$ is the diffusion coefficient, $V_H$ is the settling speed.

\end{itemize*}





\section{Modello solare.}

\subsection{Per punti}

\tool{

\begin{itemize}
\item SSM: Description of solar interior reproducing observed properties with obs. error (Phis/chim input within uncertainties).
\item Before helioseismology: initial abundances $Y_0, Z_0,\alpha\xrightarrow{SSM}\rsun{},\lsun{},Z_{\odot}^{ph}$.
\item Simmetria sferica. Inclusione rotazione nella struttura idrostatica?
\item Stellar evolution : solar model - Observed properties
\item \sout{Stellar structure theory is able to rationalize observed relation between L,M,R,T of the Sun.}
\item Expressions in terms of P, m, T, l, $X_i$.
\item Gradiente temperatura: LTE.
Lo stato termodinamico del Sole \'e di equilibrio termico locale: 
\item \sout{Fusione di leggi conservazione e trasporto energia di sezione 2?}
\item Equilibrio termodinamico locale
\item \sout{Therma equilibrium: large timescale. Thermal diffusion time} $\tau_{thermal}\approx\SI{e7}{\year}\ll\tau_n$
\item \sout{Age of the sun (solar system, Star formation (\cite{han12stellar}))}
\item \sout{Mass (\cite{ber03solar})}
\item \sout{radius}
\item \sout{Main sequence}
\item Chemical composition. Hydrogen abundance $n_H$: continuum absorption. $n_i$: line absorption.

Logaritmic abundances normalized to $n_H=\num{e12}$ particles per unit volume: $\log{A}=12+\frac{\log{n_i}}{\log{n_H}}$.
\item Solar irradiance.
\item Modello solare, parametri del modello, equazioni di base e vincoli osservativi: diagramma di HR e misure spettrometriche della fotosphera.
\item \sout{particle diffusive effect (shu pg 34)}
\item \sout{Diffusion coefficient non fa differenza tra diffusion e settling}
\item variazione composizione chimica: fusione nucleare, settling e diffusione; tempo di mixing per zone convettive (5.5.3 pg 70)
\item \sout{Instabilit\'a convettiva: caratteristiche.}
\item \sout{convective mixing (chap 5)}
\item $G\msun{}=(132712438\pm5)\SI{e12}{\cubic\meter\per\square\second}$ (high precision), lab measurement: $G=(6.672\pm0.004)\SI{e-11}{\cubic\meter\per\kilo\gram}$; $\msun{}=(1.9801\pm0.0012)\SI{e30}{\kilo\gram}$.
\item Mass loss $\dot{\msun{}}\approx\frac{\lsun{}}{c^2}+$ Solar Wind $\approx\SI{e9}{\kilo\gram\per\second}$.
\item Densit\'a: $\rsun{}=\SI{1.408}{\gram\per\cm}$.
\item g on surface: $\gsun=\frac{G\msun}{\rsun{}^2}=\SI{274}{\meter\per\square\second}$.
\item modello solare standard.
\item Approx di base per equazione di stato interno stellare (\sch{}, kippenhahn, clayton)
\item Equation of state (\cite{han12stellar})
\item Modello solare stellar modelling ( (\cite{han12stellar}))
\item sole \'e stella in sequemza principale
\item ZAMS (\cite{han12stellar})
\item \sout{MLT( (\cite{han12stellar}))}
\item Heliosismic constrain: deconvolution of helioseismic data provides the depth of convective zone $R_b$ for which there is no free parameter.(Vorontsov91, JCD gough 91, Dzi91)
\item Region of partial ionization in convective zone where the stratification is nearly adiabatic except for the top: the structure of convective zone depends ess on equation of state  and composition while not directly affected by opacity.
\item Effects of thermodynamic state and composition: $\Gamma_1$ is suppressed relative to $\frac{53}{3}$ for a full ionized gas in the zones of partial ionization of abundant elements. Determination of He abundance is in principle possible because of reduction in \gexp{} in second ionization zone of He (depends on He abundance).
\item Current solar models (93) can reproduce oscillation spectrum to \SI{+-10}{\micro\hertz} over frequency range \SIrange{1500}{4500}{\micro\hertz} is 2 order of magnitude greater than observation errors \SI{0.1}{\micro\hertz}.
\item It is evident that our ability to investigate the microphysics by means of helioseismology depends on the validity of the other assumptions on which the computations are based. In fact, the computation of standard solar models ignores, or grossly simplifies, a number of processes that might be labelled the macrophysics of the Sun.
It is assumed that there is no mixing or diffusion in the solar interior, so that the composition in any given mass-shell is determined solely by the local nuclear burning. 

With these assumptions the structure is largely determined by the microphysics of the solar interior.

\item Microphysics
\begin{itemize*}
\item Equation of state
\item Opacity
\item Nuclear reaction rate
\end{itemize*}

\item Nuclear energy sources.
CNO cycle: $X_{CN}=0.0045$ (Detailed abundances: opacity table).
Most important reactions are those of proton chain: (con $\lambda$ reaction rate)
\begin{align*}
&\dot{n_p}=-\frac{3}{2}\lambda_{pp}n_p^2+\lambda_{33}n_3^2-\lambda_{34}n_3n_4-4\lambda_{p14}n_pn_{\indices{^{14}}N}\\
&\dot{n_3}=\frac{1}{2}\lambda_{pp}n_p^2-\lambda_{33}n_3^2-\lambda_{34}n_3n_4\\
&\dot{n_4}=\frac{1}{2}\lambda_{33}n_3^2+\lambda_{34}n_3n_4+\lambda_{p14}n_pn_{\indices{^{14}}N}
\end{align*}


\item Macrophysics
\begin{itemize*}
\item energy transport
\item dynamics of convection 
\item convective overshoot 
\item microscopic diffusion 
\item core mixing 
\item magnetic fields.
\end{itemize*}


\item The connection between the physical properties of the solar plasma which we wish to probe by means of helioseismology and the observed frequencies goes through computations of solar models.

Dependence of the models, and hence the oscillation frequencies, on the microphysics, particularly the equation of state.

\item Microscopic diffusion is likely to have some effect on the composition profile in the convectively stable region, yet with a few exceptions has been ignored.

\item \sout{Depth of convection zone: $\alpha=\frac{l}{H_P}$ increases comporta aumento nella profondit\'a} della zona convettiva. Convection is capable of carrying the solar luminosity at greater depth: outer convection zone increases with alpha.
\item Energy conservation: $F_R+F_C=\frac{\lsun{}}{4\pi r^2}$.
\item Solar p mode of high degree respond in sensitive way to changes of $\alpha$.

\item Convection zone deeper (gough77, ulrich rhodes 77)
\item Frequenze sensibili ai dettagli dell'equazione di stato (Berthomieu80, lubow80)
\item Convection zone depth approx \SI{200000}{\kilo\meter}.
\item Dalsgaard85: base of convetion zone as change in curvture of $c(r)$.
\item \sout{Una maggiore effecienza del trasporto convettivo di energia si riflette in una} minore differenza tra il gradiente di temperature adiabatico e effettivo. L'eccesso di entropia specifica rispetto allo stato marginale
\begin{equation}
    \Delta S=\int c_P(\nabla-\nabla_a)\,d\ln{P}
\end{equation}
decreases with alpha: it's the entropy that physically characterizes convective solar envelope.
\item I modelli solari prefedono frequenze troppo alte per modi p di alto l: solar convection theory overestimate $\Delta S$ (jump of entropy across outer layers).
\item Discepancy in radiative interior S/M: deficiency in opacity.
\item Need for closure relation.

Need for closure relation between P and $\rho$:  energetic, first law of thermodynamics.
Relazioni politrope $P=\rho\expy{\gamma}$.

\item Meridional circulation, turbolent mixing.

Proffitt, Michaud (1991) - bahcall (1990) - Michaud, Vauclair (1991) - Elsworth (1990)
\item Noble gases  don't have line in photospheric spectrum: large excitation energy.
\item Spectroscopic data: $\frac{Z_s}{X_s}\approx0.023$.
\item Collision cross section in fully ionized plasma (heuristic): $l^2=(\frac{e_1e_2}{KT})^2$. $\sigma\approx l^2$.
A particle is accelerated by gravity between randomizing collisions. Mean downward velocity $v_{Drift}=g\tau$ where $\tau=(\sigma n v_{Th})^{-1}$ is the mean time between collisions, n numerical density of scattering, $v_{Th}$ mean thermal velocity $\sqrt{\frac{KT}{m}}$: $v_{Drift}=g(KT)\expy{\frac{3}{2}}m\expy{\frac{1}{2}}(e_1e_2)\expy{
-2}n\expy{-1}$.
\item Diffusion speed up as star ages.
\item Element diffusion: Bahcall (93), Flow equation: Burges (69), Diffusion velocities for H, He, O, Fe(Bahcall leob 90).
\item Burges (60):
\begin{equation*}
\PDy{t}{\rho_s}+\nabla\cdot(\rho_s\vec{w_s})=(\PDy{t}{\rho_s})_{Nuc}    
\end{equation*}

\item He diffusion in the sun: Bahcall pinsonneault, Noerdlinger76.
\item Element diffusion is drive in stars by pressure gradient (or gravity), temperature gradient, composition gradient, and radiation pressure. Using BL90 dimensionless variables the mass fraction of element s satisfy the equation
\begin{equation*}
    \PDy{t}{X_s}=-\frac{1}{\rho r^2}\PDof{r}[r^2X_sT\expy{\frac{5}{3}}\xi_s(r)]
\end{equation*}
and $\xi_s$ is related to diffusion velocity of specie s: $\xi_s(r)=\frac{w_s(r)\rho(r)}{T\expy{\frac{5}{2}}(r)}$.
\begin{equation*}
    \xi_s(r)=A_P(s)\PDy{r}{\ln{P}}+A_T(s)\PDy{r}{\ln{T}}+A_H(s)\PDy{r}{\ln{C_H}}
\end{equation*}
la concentrazione della specie s \'e $C_s=\frac{\frac{X_s}{A_s}}{\sum
_i\frac{Z_iX_i}{A_i}}$.

\item Effects of diffusion and settling are usually included  in SM calculations: the rate of change of hydrogen abundance is written

\begin{align*}
\PDy{t}{X}=R_H+\frac{1}{r^2\rho}\PDof{r}[r^2\rho(D_H\PDy{r}{X}+V_HX)]
\end{align*}

\end{itemize}
}


La luminosit\'a e la temperatura efficace sono le coordinate nel diagramma di \hr{} di una stella da cui \'e possibile stimare il raggio, trasporto di energia, rotazione media e altre caratteristiche che caratterizzano una regione del diagramma \hr{}. La massa, l'et\'a e la composizione chimica sono le grandezze di partenza di un modello stellare che riproduca le caratteristiche di L e $T_e$.

La determinazione della struttura solare sulla base delle equazioni fondamentali dell'equilibrio stellare permette anche di calcolare accuratamente le frequenze adiabatiche relative al dato modello solare: dalle discrepanze fra le frequenze osservate e quelle calcolate \'e possibile determinare carenze nella fisica o nelle semplificazioni del modello.


La struttura interna di una stella \'e deteminata da leggi di conservazione (massa, quantit\'a di moto, energia) e dall'equazioni che caratterizzano lo stato del gas nell'interno solare a $(T,\rho)$ date.

\subsection{Conservazione dell'energia.}

La prima legge della termodinamica esprime la conservazione dell'energia interna (per unit\'a di volume)
\begin{equation*}
\TDy{t}{q}=\TDy{t}{E}+P\TDof{t}(\frac{1}{\rho})=0=\TDy{t}{E}+P\TDy{t}{V}
\end{equation*}
e le equazioni equivalenti utilizzando gli esponenti adiabatici $\Gamma_i$

\begin{align*}
&\Gamma_1=\Dcvar{\TDy{\ln{\rho}}{\ln{P}}}{Ad}=\gamma_{ad}&\intertext{$\Gamma_2$ \'e definito da: }\\
&\frac{\Gamma_2}{\Gamma_2-1}=\Dcvar{\TDy{\ln{T}}{\ln{P}}}{Ad}=\frac{1}{\nad}\\
&\Gamma_3=\Dcvar{\TDly{\rho}{T}}{ad}+1\\
&\frac{\Gamma_1}{\Gamma_3-1}=\frac{\Gamma_2}{\Gamma_2-1}
\end{align*}


\begin{align}
&\TDy{t}{\ln{T}}=\frac{\Gamma_2-1}{\Gamma_2}\TDy{t}{\ln{P}}+\frac{\TDy{t}{q}}{c_PT}\\
&\TDy{t}{\ln{P}}=\Gamma_1\TDy{t}{\ln{\rho}}+\frac{\rho(\Gamma_3-1)}{P}\TDy{t}{q}
\end{align}

Per un gas perfetto $\gamma=\frac{c_P}{c_v}=\Gamma_i$.



Detta $W=E_i+E_g$ dal teorema del viriale e dalla conservazione dell'energia segue
\begin{equation}
L=-\frac{1}{2}\dot{E_g}=\dot{E_i},\ \TDy{t}{W}+L=0
\end{equation}

Il tempo caratteristico per una stella (di massa solare) in cui il termine gravitazionale \'e dominante \'e $\tkh{}=\frac{E_g}{L}\approx\frac{U}{L}\approx\frac{GM^2}{2RL}\approx\SI{1.6e7}{\year}$.

La fonte delle luminosi\'a solare sono i processi nucleari che avvengono nella parte centrale, in particolare le reazioni del ciclo $\Pproton\Pproton$ forniscono il $99.9\%$ dell'energia

\begin{equation}
\TDy{t}{q}=\epsilon-\frac{1}{\rho}\scap{\nabla}{F}
\end{equation}
implica che
\begin{align*}
&\TDy{r}{L}=4\pi r^2[\rho\epsilon-\rho\TDof{t}\frac{u}{\rho}+\frac{P}{\rho}\TDy{t}{\rho}]&\intertext{Nel caso stazionario:}\\
&\TDy{t}{q}=0\ \Rightarrow\ dL=4\pi\rho\epsilon\,dr&\intertext{Il tempo trascorso da una stella simile al sole in sequenza principale \'e (fusione tutto H in He)}\\
&\tau_n\approx\frac{E_n}{L}=\frac{Q\msun}{\lsun}\approx\SI{e+11}{\year},\\ &Q=\SI{6.3e18}{\erg\per\gram}
\end{align*}

In assenza di diffusione la composizione chimica \'e alterata solo dai processi di fusione.

L'energia interna specifica \'e determinata dall'energia cinetica delle particelle libere e dall'energia di ionizzazione

\begin{align}%% Uso E al posto di U e L (epton) al posto di E
&E(T,P)=\frac{3RT}{2\mu}+\frac{1}{\rho}[n_{H^+}\chi_H+n_{He^+}\chi_{He}+n_{He^{++}}(\chi_{He}+\chi_{He^+}]
\end{align}

mentre il peso molecolare medio

\begin{align}
&\mu=\frac{\mu_0}{1+L}&\intertext{$E$ rappresenta il numero di elettroni liberati dalla ionizzazione di H e He diviso il numero delle altre particelle.}\\
&\mu_0=\frac{1}{X+Y/4+Z/2}\\
&L=\mu_0[\eta_HX+(\eta_{He}+2\eta_{He^+})Y/4]&\intertext{$\eta_A$ indica il grado di ionizzazione della specie A: numero atomi ionizzati sul totale della specie A.}\nonumber
\end{align}

\subsection{Adiabatic processes}

Most of the gas in a star can be thought as adiabatic: any process that take place on a timescale shorter than $\tkh{}$ can be thought of as adiabatic.

\begin{align*}
&\TDy{t}{u}+P\TDof{t}\frac{1}{\rho}=\epsilon-\TDy{m}{F}=0&\intertext{and for many types of gas $u=\phi\frac{P}{\rho}$ quindi}\\
&\frac{dP}{P}=(\frac{\phi+1}{\phi})\frac{d\rho}{\rho},\ \ln{P}=\gamma_a\ln{\rho}+\ln{K_a}&\intertext{the constant $K_a$ is determined by the entropy of the gas}
\end{align*}

All monoatomic ideal gas have $\gamma_a=\frac{5}{3}$, all relativistic gasses have $\gamma_a=\frac{4}{3}$.

Nel caso $K_a$ sia costante, tipo nella zone convettive, posso usare una relazione politropica con $\gamma_P=\gamma_a$.

\begin{usefull}{Difference polytropic relation vs adiabatic exponent}

\begin{itemize}
\item Polytropic relation describes how pressure change with density inside as one moves through the star.
\item Adiabatic equation of state describe how how a given gas shell would respond to being compressed.
\end{itemize}

\end{usefull}


\subsection{Trasporto dell'energia.}

Per ricavare il gradiente di temperatura considero il momento trasferito dai fotoni ad un elemento di gas di volume infinitesimo $dSdr$:

\begin{align}
&n_{\nu}\,dS\,c*(\kappa_{\nu}\rho\,dr)&\intertext{\'e il numero di fotoni di frequenza $\nu$ assorbiti dall'elemento di gas per unit\'a di tempo,  dove $\kappa_{\nu}\rho$ \'e la probabilit\'a di assorbimento di un fotone di frequenza $\nu$ per unit\'a di lunghezza. La variazione del flusso di fotoni \'e legata al gradiente della pressione radiativa tramite (integrando su tutte le frequenze)}\nonumber\\
&dP_{rad}=-\intzi{}\,d\nu\,\frac{h\nu}{c}n_{\nu}c*(\kappa_{\nu}\rho\,dr)&\intertext{utilizzando l'ipotesi di equilibrio termodinamico locale}\\
&\TDof{r}(\frac{1}{3}aT^4)=-\intzi{}\frac{n_{\nu}ch\nu}{c}\kappa_{\nu}\rho&\intertext{con}\nonumber\\
&n_{\nu}=\frac{8\pi}{c^3}\frac{\nu^2}{\expy{\frac{h\nu}{kT}}-1}\\
&a=\SI{7.5657e-15}{\erg\per\cubic\cm\per\kelvin\tothe{4}}
\end{align}


La teoria della mixing length fornisce una spiegazione fenomenologica della convezione.

La distanza che un elemento di gas percorre prima di dissolversi \'e detta mixing length l, nella formulazione di Bohm-Vitense (1958) $l=\alpha H_P$ dove $H_P=-\frac{d\ln{P}}{dr}$ \'e l'altezza scala di pressione.

In un strato instabile per convezione ho un moto di ampiezza crescente secondo

\begin{equation}
    \PtwoDy{t}{r}=-g\frac{\Delta\rho}{\rho}=g\delta\frac{\Delta T}{T}
\end{equation}

abbiamo assunto  $\Delta P=0$ (equilibrio di pressione fra l'elemento di fluido e l'ambiente), il fattore 

\begin{equation}
    \delta=-\Dcvar{\PDly{T}{\rho}}{P}=1-\Dcvar{\PDly{T}{\mu}}{P}
\end{equation}

tiene conto delle variazioni di $\mu$, mentre $\Delta T$ \'e la differenza di temperatura tra l'elemento di fluido e l'ambiente:
\begin{equation}
    \Delta T=[(\TDy{r}{T})'-\TDy{r}{T}]\,\delta r=(\nabla-\nad{})\frac{T\,\delta r}{H_P}
\end{equation}

Una maggiore effecienza del trasporto convettivo di energia si riflette in una minore differenza tra il gradiente di temperature adiabatico ed effettivo.

L'entropia caratterizza fisicamente la zona convettiva del Sole: l'eccesso di entropia specifica rispetto allo stato marginale
\begin{equation}
    \Delta S=\int c_P(\nabla-\nabla_a)\,d\ln{P}
\end{equation}
diminuisce con $\alpha$.

Velocit\'a tipiche sono dell'ordine di centinaia di \si{\meter\per\second}: in un gas la cui viscosit\'a \'e trascurabile i moti sono turbolenti.


Nel caso di equilibrio radiativo:

\begin{align}
&\frac{dT}{T}=\frac{1}{4}\frac{\kappa\eta}{\overline{\kappa\eta(r)}}\frac{dP}{P}\\
&\eta=\frac{l(r)}{m(r)}/(\frac{L}{M})
\end{align}

Descrivo le caratteristiche del trasporto di energia verso la superficie attraverso la relazione
\begin{align*}
&\TDy{r}{T}=\nabla\frac{T}{p}\TDy{r}{p}\\
&\nabla=\TDly{P}{T}&\intu{\'e determinato dalle caratteristiche del trasporto di energia}\\
&\nabla_{Rad}=\frac{3}{16\pi a cG}\frac{\kappa P}{T^4}\frac{L(r)}{m(r)}&\intu{in equilibrio radiativo}\\
&\nabla\approx\nad{}=\Gamma_1&\intu{nelle zone convettivamente instabili.}
\end{align*}

\begin{todo}{\sch{} recipt}
Si determina in ogni strato della stella il gradiente di pressione dall'equilibrio idrostatico, il gradiente di temperatura dall'equilibrio radiativo.

se la condizion di stabilit\'a
\begin{equation}
    -(1-\frac{1}{\gamma})\frac{T}{P}\TDy{r}{P}>-\TDy{r}{T}
\end{equation}
\'e verificata non si ha convezione e il gradiente di temperatura \'e quello radiativo, altrimenti siamo in presenza di convezione e il gradiente di temperatura \'e approssimativamente adiabaatico.

\end{todo}

La zona convettiva occupa il $30\%$ pi\'u esterno del raggio solare infatti le basse temperature causano un aumento dell'opacit\'a e il gradiente termico necessario per il trasportare la luminosit\'a solare \'e superiore al gradiente adiabatico. In questa regione i moti convettivi assicurano l'omogeneit\'a chimica.

\subsection{Stabilit\'a convettiva}


Considero sotto quali condizioni una perturbazione radiale infinitesima di un elemento di fluido cresce esponenzialmente a causa della forza di galleggiamento 
\begin{equation}
\rho\PtwoDy{t}{(\Delta r)}=-g\Delta\rho=-g[\Dcvar{\TDy{r}{\rho}}{e}-\Dcvar{\TDy{r}{\rho}}{amb}]\Delta r
\end{equation}
La forza di archimede ha direzione opposta alla perturbazione se $\Delta\rho>0$.

Considero un'equazione di stato generica $\rho(P,T,\mu)$ e definisco i coefficienti $\alpha,\beta,\delta$ tramite:
\begin{equation}
\frac{d\rho}{\rho}=\alpha\frac{dP}{P}-\delta\frac{dT}{T}+\phi\frac{d\mu}{\mu}\label{eq:deltatherm}
\end{equation}

Riscrivo l'equazione del moto, considerando il moto dell'elemento in equilibrio di pressione con l'ambiente:

\begin{equation}
\PtwoDy{t}{(\Delta r)}=-g\frac{\delta}{H_P}[\nabla_e-\nabla-\frac{\phi}{\delta}\nmu{}]\Delta r\label{eq:galleggiamento}
\end{equation}

dove ho introdotto i gradienti termici per il blob e ambientale e il gradiente di composizione chimica nella forma

\begin{align}
&\nabla=\Dcvar{\TDly{P}{T}}{amb},\ \nabla_e=\Dcvar{\TDly{P}{T}}{blob},\ \nmu{}=\Dcvar{\TDly{P}{\mu}}{amb}&\intertext{e definisco le lunghezze caratteristiche per variazione di densit\'a e pressione:}\\
&\densityscale{}=-\frac{dr}{d\ln{\rho}},\ H_P=-\frac{dr}{dP}
\end{align}

Suppongo adesso un moto del blob adiabatico $\nabla_e=\nabla_{ad}=\frac{P\delta}{T\rho c_P}$ e introduco la frequenza di \bv{}:
\begin{align}
&N^2=g(\frac{1}{\Gamma_1P}\TDy{r}{P}-\frac{1}{\rho}\TDy{r}{\rho})\label{eq:bvfs}\\
&N^2=g(\frac{1}{\densityscale{}}-\frac{g}{c_s^2})\label{eq:bvfsdensita}
\end{align}
rappresenta la massima frequenza sotto cui pu\'o oscillare una particella di fluido sottoposta a onde di gravit\'a mantenendo l'equilibrio di pressione con l'ambiente.

La variazione di composizione ambientale pu\'o essere descritta tramite $\densityscale{}$ in \eqref{eq:bvfsdensita}, quindi riscrivo l'equazione \eqref{eq:galleggiamento}
\begin{equation}
\PtwoDy{t}{(\Delta r)}=-N^2\Delta r
\end{equation}
che descrive un comportamento oscillatorio per $N^2>0$, cio\'e uno strato di gas del Sole \'e stabile per convezione se

\begin{align}
&\nrad{}<\nad+\frac{\phi}{\delta}\nmu{}\label{eq:ledoux}&\intertext{dove ho usato $\nabla_{amb}=\nrad{}$ definito in \eqref{eq:radiativegradient}, cio\'e il gradiente che si ha nel caso la luminosit\'a si trasportata dai fotoni}\nonumber
\end{align}

La zona convettiva occupa il $29\%$ pi\'u esterno del raggio solare e il $2\%$ della massa: questa regione \'e chimicamente omogenea. Le basse temperature causano un aumento dell'opacit\'a e il gradiente termico necessario per trasportare la luminosit\'a solare \'e superiore al gradiente adiabatico, il cui valore \'e diminuito dal calore latente dell'idrogeno solo parzialmente ionizzato.

Una maggiore efficienza del trasporto convettivo di energia si riflette in una minore differenza tra il gradiente di temperature adiabatico ed effettivo: per determinare lo scostamento dalla stratificazione adiabatica dovuto alle perdite radiative utilizzo la teoria della mixing-length.

\begin{todo}{Convective flux}
Il flusso di energia convettivo \'e $F_C=\alpha\rho c_P\exv{v}T\frac{\nabla-\nabla_e}{2}$, $\nabla$ \'e determinato tramite l'equazione

\begin{align}
\frac{9}{8U}(x-U)^3+x^2-U^2-\nrad{}+\nad{}=0\label{eq:mixingcubic}&\intertext{con $U$ che collega le caratteristiche del moto del blob con le perdite radiative e $x=\sqrt{\nabla-\nad{}+U^2}$:}\nonumber\\
U=\frac{3acT^3}{c_P\rho^2\kappa l_m^2}\sqrt{\frac{8H_P}{g\delta}}\propto\frac{\tau_{ff}}{\tau_{rad}}
\end{align}


\end{todo}

\subsection{Vibrational stability}

In a dynamical stable layer an oscillating mass has in general $\Delta T\neq0$: if $\Delta T>0$ it will lose heat to its surrounding by radiation, if $\Delta T<0$ it will gain heat.

This means it will not move adiabatically.

For small deviation from adiabaticity, thermal adjustment time $\tau_{adj}\gg\Pi$, the temperature excess of element is

\begin{align*}
&\Delta T=[\Dcvar{\TDy{r}{T}}{e}-\Dcvar{\TDy{r}{T}}{s}]\Delta r\\
&=-\frac{T}{H_P}(\nabla_e-\nabla)\Delta r
\end{align*}

Dynamical stability means $\frac{\Delta\rho}{\Delta r}>0$: if the layer is chemically homogeneous then stability criterion is $\nabla_e-\nabla>0$ then $\Delta T<0$ for $\Delta r>0$: above its equilibrium position the element is cooler than the surroundings and receive energy by radiation, this reduces $\nabla_e-\nabla$, $\Delta\rho$ and the restoring force such that the element is less accelerated back to its original position. This result in oscillation of decreasing amplitude.This radiative damping shows up as a small positive immaginary part in $\omega$, the oscillatory part is still very close to adiabat.

In case of inhomogeneous layers for $\nmu{}$ big enough we can have $\nabla_e-\nabla<0$ then $\Delta T>0$ for $\Delta r>0$ and lifted element will radiate into surroundings: this increases $\nabla_e-\nabla$, $\Delta\rho$ and the restoring force and the element will oscillate with slowly increasing amplitude.

\begin{usefull}{Overstability or vibrational instability}
The growing oscillation may leads to chemical mixing of elements and surroundings thus destroying stabilizing effect of $\nmu{}$.
The reaction of other layers of the star may provide enough damping to suppress the overstability
\end{usefull}


\subsection{Entropy jump in cobection zone}

La luminosit\'a dipende fortemente dal valore di $Y_0$,  mentre il raggio da $\alpha$, parametro che regola l'efficienza del trasporto convettivo nella regione esterna caratterizzata fisicamente dall'entropia il cui valore \'e determinato, a meno di una costante additiva, dalla zona superadiabatica vicino alla superficie. Infatti, in un modello semplificato in cui si descrive la zona convettiva con stratificazione quasi-adiabatica tramite $P=K\rho\expy{\gamma}$, l'eccesso di entropia specifica tra la fotosfera e la parte quasi-adiabatica
\begin{equation}
\Delta S=\int_{\ln{P_{Ph}}}^{\ln{P^*}} c_P(\nabla-\nabla_a)\,d\ln{P}
\end{equation}
con $P^*$ tale che $\nabla-\nad{}\ll1$, parametrizza la variazione di $\rsun{}$ e della profondit\'a della zona convettiva, essendo $\delta\ln{K}\approx\frac{5}{3}\frac{\delta(\Delta s)}{c_P}$:
\begin{align}
&\frac{\delta R}{R}\approx 0.24\,\delta\ln{K}\approx-0.24\,\delta\ln{\alpha}\\
&\frac{\delta d_{cz}}{R}\approx-0.02\,\delta\ln{K}&\intertext{cio\'e la calibrazione del raggio influenza poco la profondit\'a della zona convettiva.}\nonumber
\end{align}
\cite{chr97effects}
Scelgo $Y_0$ e $\alpha$ che forniscono luminosit\'a e raggio $\rsun{}=\SI{6.96e8}{\meter}$ attuali del Sole: risulta $\frac{R_b}{\rsun{}}\approx0.710$ e il valore di $Y_0=0.250$.

\subsection{Thermal adjustment time}

\begin{definition}{Thermal adjustment time}
A dynamical instability grows on time-scale given by $\sqrt{\frac{H_P}{g}}$.

\end{definition}

Let's consider a mass element with $\Delta T>0$: superposed onto radial flux $\vec{F}$ there will be a local non radial flux $\vec{f}$ carrying the surplus of energy of the element to its surroundings.

\begin{align*}
&f=\frac{4acT^3}{3\kappa\rho}|\PDy{n}{T}|\\
&\PDy{n}{T}\approx\frac{2\Delta T}{d}&\intertext{the radiative loss per unit time from whole surface of the blob is}\\
&\lambda=Sf\\
&\rho Vc_P\TDy{t}{T_e}=\rho Vc_P\TDy{t}{\Delta T}=-\lambda\\
&\PDy{t}{(\Delta T)}=-\frac{\Delta T}{\tau_{adj}}\\
&\tau_{adj}=\frac{\kappa\rho^2c_Pd^2}{16acT^3}=\frac{\rho Vc_P\Delta T}{\lambda}
\end{align*}

For large elements that are far enough from region of marginal stability one has $\tau_{adj}\gg\frac{1}{\omega_{ad}}$.

\subsection{Secular instability}

Consider a blob situated in surroundings somewhat different: \mblock{\Delta\mu\neq0,\ \nmu{}=0}, for example if two homogeneous layers of different composition are obove each other and a blob from one layer is displaced into the other. The blob is supposed to be in mechanical equilibrium with its surroundings \mblock{\Delta\rho=\Delta P=0} and
\begin{equation}
\delta\frac{\Delta T}{T}=\phi\frac{\Delta\mu}{\mu}\label{eq:secularmu}
\end{equation}
for $\Delta\mu>0$ the blob is hotter and radiates toward surroundings: the loss of energy under pressure balance leads to an increased density and the blob sinks until $\Delta\rho=0$; $\Delta\mu$ is unchanged $\Delta T>0$ as before etc.

The blob slowly sink or rise with a velocity $v_{\mu}$ such that $\Delta T$ remains constant according to \eqref{eq:secularmu}
\begin{align*}
&\frac{1}{T}\PDof{t}(\Delta T)=(\nad{}\PDy{t}{\ln{P}}-\frac{\Delta T}{T\tau_{adj}})-\nabla\PDy{t}{\ln{P}}&\intertext{il secondo termine tra parentesi rappresenta gli scambi radiativi, il primo i cambi di temperatura per compressione/espansione adiabatica}\\
&\PDy{t}{\ln{P}}=-\frac{v_{\mu}}{H_P}
\end{align*}

Since $\Delta\mu$ doesn't varies if blob moves in chem homogeneous layer we have $\Delta T=0$ and we can solve previus equations for the velocity
\begin{equation}
v_{\mu}=-\frac{H_P}{(\nad{}-\nabla)\tau_{adj}}\frac{\phi}{\delta}\frac{\Delta\mu}{\mu}
\end{equation}

\subsection{Parametri iniziali.}

\begin{todo}{Calibrazione modello solare.}
Il Sole \'e la stella pi\'u vicina ed ha massa $\msun=$\SI{1.989e33}{\gram}, raggio $\rsun=$\SI{6.9599e10}{\cm}, luminosit\'a $\lsun=$\SI{3.846e33}{\erg\per\second} ed et\'a (in sequenza principale) $\tsun=$\SI{4.566+-0.005e9}{\year}.


La determinazione dell'abbondanza iniziale di He \'e soggetta a grande incertezza se determinata tramite le righe di assorbimento dell'atmosfera quindi \'e usata come parametro.

Oltre all'equazione di stato che esprime $U(P,T)$ per la detrminazione della struttura solare sono necessari $\epsilon(\rho,T)$ e $\kappa(\rho,T)$.
\end{todo}

Un modello stellare per il Sole attuale si ottiene integrando numericamente le equazioni fondamentali della struttura stellare a partire dalle condizioni al bordo. L'incertezza sull'abbondanza iniziale di He e sulla profondit\'a della zona convettiva rendono necessaria una calibrazione in funzione della luminosit\'a e raggio attuali. L'et\'a del sistema solare \'e nota e tramite modelli di formazione e analisi meteoriti si ha $t_{\odot}=\SI{4.9+-0.1e9}{\year}$ con incertezza dovuta al periodo di solidificazione dei meteoriti.

La luminosit\'a dipende fortemente dal valore di $Y_0$, scelgo quello che fornisce la luminosit\'a attuale $L=\SI{3.845e26}{\watt}$, mentre il raggio $\rsun{}=\SI{6.96e8}{\meter}$ dal rapporto fra mixing lenght e lunghezza-scala di pressione $\alpha=\frac{l}{H_P}$, $H_P=-\frac{1}{\TDy{r}{\ln{P}}}$, parametro che regola l'efficienza del trasporto convettivo nella regione esterna.

\begin{align*}
&\ln{L}=\ln{\lsun{}}+a(Y_0-Y_{0\odot})+b(\alpha-\alpha_{\odot})\\
&\ln{r}=\ln{\rsun{}}+c(Y_0-Y_{0\odot})+d(\alpha-\alpha_{\odot})\\
&a=\PDy{Y_0}{\ln{L}}=8.6,\ b=\PDy{\alpha}{\ln{L}}=0.02\\
&c=\PDy{Y_0}{\ln{r}}=2.1,\ d=\PDy{\alpha}{\ln{r}}=-0.19
\end{align*}

La zona convettiva risulta di \SI{200000}{\kilo\meter} e il valore di $Y_0=0.256$.

\begin{figure}[!ht]
\centering
\includegraphics[width=(\textwidth),height=(\textheight-11mm),keepaspectratio]{SchemSstructure}
\caption{struttura schematica sole}
\end{figure}

La luminosit\'a dipende fortemente dal valore di $Y_0$,  mentre il raggio da $\alpha$, parametro che regola l'efficienza del trasporto convettivo nella regione esterna caratterizzata fisicamente dall'entropia il cui valore \'e determinato, a meno di una costante additiva, dalla zona superadiabatica vicino alla superficie. Infatti, in un modello semplificato in cui si descrive la zona convettiva con stratificazione quasi-adiabatica tramite $P=K\rho\expy{\gamma}$, l'eccesso di entropia specifica tra la fotosfera e la parte quasi-adiabatica
\begin{equation}
\Delta S=\int_{\ln{P_{Ph}}}^{\ln{P^*}} c_P(\nabla-\nabla_a)\,d\ln{P}
\end{equation}
con $P^*$ tale che $\nabla-\nad{}\ll1$, parametrizza la variazione di $\rsun{}$ e della profondit\'a della zona convettiva, essendo $\delta\ln{K}\approx\frac{5}{3}\frac{\delta(\Delta s)}{c_P}$:
\begin{align}
&\frac{\delta R}{R}\approx 0.24\,\delta\ln{K}\approx-0.24\,\delta\ln{\alpha}\\
&\frac{\delta d_{cz}}{R}\approx-0.02\,\delta\ln{K}&\intertext{cio\'e la calibrazione del raggio influenza poco la profondit\'a della zona convettiva.}\nonumber
\end{align}
%\cite{chr97effects}
Scelgo $Y_0$ e $\alpha$ che forniscono luminosit\'a e raggio $\rsun{}=\SI{6.96e8}{\meter}$ attuali del Sole: risulta $\frac{R_b}{\rsun{}}\approx0.710$ e il valore di $Y_0=0.250$.

\clearpage

\subsection{Mean free path of plasma particles}

\cite{pit12kinetics}

\begin{definition}{Almost ideal plasma}
Suffiently rarefied to apply equation of transport to it.
\begin{equation*}
kT\gg \frac{e^2}{\overline{r}}\approx e^2n\expy{\frac{1}{3}}
\end{equation*}
\end{definition}

\begin{definition}{Debye length}
\begin{equation*}
\frac{1}{a^2}=(\frac{4\pi}{T})\sum_an_a(z_ae)^2
\end{equation*}
\end{definition}

Pg .184

\begin{usefull}{Ion-ion mean free path}
Ion-Ion collision mean free path
\begin{align*}
&l_i\approx \frac{T_i^2}{4\pi e^4 n L_i}\\
&L_i=\log{\frac{aT_i}{e^2}}&\intu{Coulomb logarithm\index{Coulomb logarithm}}\\
\end{align*}
\end{usefull}


\begin{usefull}{Transport coefficients}
Using kinetic theory of gas
\begin{itemize}
\item Electrical conductivity $\sigma$. 
\begin{align*}
&v\approx\tau e\frac{E}{m},\ j\approx env\\
&\sigma\approx\frac{e^2n\tau}{m}\approx\frac{e^2nl}{mv_T}\\
&\sigma\approx \frac{T_e\expy{\frac{3}{2}}}{e^2m\expy{\frac{1}{2}}L_e}\\
\end{align*}

\item Therma conductivity: electron play the main part.
\begin{align*}
&\kappa\approx n_el_ev_{T_e}c_e\, c_e\approx1
&\kappa\approx\frac{T_e\expy{\frac{5}{2}}}{e^4m\expy{\frac{1}{2}}L_e}
\end{align*}

\item the viscosity is mainly due to ions since they carry most of the momentum; moreover is almost unchanged after collision with electron. we consider ii only:
\begin{align*}
&\eta\approx n_iMl_iv_{T_i}\\
&\eta\approx M\expy{\frac{1}{2}}\frac{T_e\expy{\frac{5}{2}}}{e^4L_i}
\end{align*}
\end{itemize}

\end{usefull}

\subsection{Processi di diffusione.}

Nei modelli solari standard i processi di diffusione non sono generalmente inclusi ma modelli con diffusione forniscono risultati pi\'u aderenti alle frequenze osservate.


I processi di diffusione inglobano diversi effetti: la gravit\'a tende a concentrare gli elementi pi\'u pesanti verso il centro, il campo elettrico mantiene gli elettroni ancorati ai nuclei, la diffusione termica concentra le particelle pi\'u cariche e pi\'u pesanti nelle zone pi\'u calde, mentre la diffusione proporzionale al gradiente di concentrazione $C_s=\frac{n_s}{n_e}$ diminuisce le disomogeneit\'a.

Definisco il parametro di plasma per specie s,t:
\begin{align}
&\Lambda_{st}=\frac{3KTr_D}{|e_se_t|}\\
&r_D=\sqrt{\frac{KT}{4\pi\sum_sn_se_s^2}}
\end{align}
che indica il grado di interazione tra le due specie.


\begin{todo}{Inhomogeneities nuil up base convection zone.}
In models that incorporate the diffusion and
gravitational settling of helium and heavy-elements, the abundances of these
elements build up below the convection-zone base.
\end{todo}

Processi di diffusione modificano l'abbondanza degli elementi, il peso molecolare medio e l'opacit\'a. Sebbene il tempo caratteristico per percorre un raggio solare si relativamente lungo $\tau_{diff}\approx\SI{6e13}{\year}$ i processi di diffusione sono apprezzabili rispetto SSM con/senza, in particolare:

\begin{itemize}
    \item Diminuzione della profondit\'a della zona convettiva circa $2\%$.
    \item Abbondanza He iniziale $+0.4\%$ nei modelli con diffusione.
    \item La diminuzione di He rispetto ad H comporta un'aumento delle stime di Z rispetto ad H del $3\%$.
\end{itemize}

Le differenze nella struttura interna hanno effetto sulle frequenze di oscillazione predette per circa \SIrange{1}{5}{\micro\hertz}.

\begin{figure}[!ht]
\centering
\includegraphics[width=(\textwidth),height=(\textheight-11mm),keepaspectratio]{diffusionDnu}
\caption{Differneza nelle frequenze previste.}
\end{figure}

Differenze nelle frequenze calcolate per modelli con/senza diffusione:
\begin{itemize}
    \item Effetto della diffusione sui modi p \'e proporzionale al coefficiente di diffusione.
    \item Aumente frequenze modi p di basso grado nell'intero range \SIrange{1000}{4500}{\micro\hertz}.
    \item Diminuisce di \SIrange{1}{5}{\micro\hertz} le frequenze dei modi p di alto grado.
    \item Le differenze nei modi p di grado intermedio son determinate dal grado di penetrazione oltre il fondo della zona convettiva.
\end{itemize}

\clearpage

\subsection{Helioseismically constrained model}

\subsubsection{Helioseismology, solar models and NF (innocenti97)}

Il valore dell atemperatura centrale \'e determinato da opacit\'a $\kappa$ e $\frac{Z}{X}$: we allow that both are rescaled by multiplicative factor with respect to value used in SSM, these scaling factors then determined by helioseismic constraints on convective envelope.

\subsubsection{Solar model from helioseismology (Dziembowski90)}

Having determined $P(r)$ and $\rho(r)$ in the Sun's interior we can attemnpt to construct a comlpete helioseismologica model assuming

\begin{itemize}
    \item thermal equilibrium
    \item $T(\rho,P,X)$
    \item opacity in the form $\kappa(\rho,T,X)$
    \item Nuclear energy generation rate $\epsilon(\rho,T,X)$.
    
    But in outer part of solar core $He\indices{^3}$ content cannot be determinedfrom equilibrium condition.
\end{itemize}


\chapter{Equazione di stato}
\PartialToc

Le correzioni alle grandezze termodinamiche si esprimono tramite $x=\frac{l_L}{r_D}$ con

\begin{equation*}
r_D=\sqrt{\frac{KT}{4\pi\sum_sn_se_s^2}},\ l_L=\frac{e^2}{KT}
\end{equation*}

e in particolare la correzione alla pressione risulta negativa:

\begin{align*}
&P_{ES}=\frac{1}{3}U_{ES}<0\shortintertext{con}\nonumber\\
&U_{ES}=\sum eZ\overline{n}_ZV_{ES}=-\frac{e^3(\sum Z^2\overline{n}_Z)\expy{\frac{3}{2}}}{2(4\pi\epsilon_0)(\epsilon_0KT)\expy{\frac{1}{2}}}
\end{align*}

Il contributo degli elettroni, detta $n_e$ la densit\'a numerica, $\psi=\frac{KT}{KT_F}\approx\num{3e-6}T(\frac{\mu_e}{\rho})\expy{\frac{2}{3}}$ il parametro di degenerazione e $u_k$ energia cinetica dell'elettrone, \'e determinato da
\begin{align}
&\rho N_A\frac{1+X}{2}=\intzi{}\frac{8\pi p^2\,dp}{h^3(\exp{\frac{u_k}{KT}-\psi}+1)},\ \beta P-\rho\gasconstant{}(X+\frac{Y}{4}+\frac{Z}{\exv{A_Z}})=\frac{1}{3}\intzi{}pn_e\TDy{p}{u_k}\,dp\shortintertext{dove ho introdotto il peso atomico medio per elettrone libero (ionizzato) $\mu_e$ con}\nonumber
&\frac{1}{\mu_e}\approx X+\frac{1}{2}Y+\frac{1}{2}(1-X-Y)=\frac{1+X}{2}
\end{align}

\chapter{Oscillazioni lineari adiabatiche. Modi di oscillazione.}
\PartialToc


\section{Per punti.}

\tool{
\begin{itemize}
    
    \item Perturbazioni: reazione del sistema: oscillazioni.
    
    \item Relazioni perturbazioni vs Langrangiana (Tolsoy): in entrambe introduco la perturbazione della posizione $\Lvar{\vec{\xi}}$, ma tolsoy gi\'a ignora la perturbazione di $g$.
    \item Adiabatic approximation: dalsgaard, stellar oscillation pg.47: why we can neglect heating term in energy equation.
    
    \item Nel caso di onde puramente acustiche
    \begin{align*}
    &\PtwoDy{t}{\rho'}=-v_S^2\nabla^2\rho'\intertext{equazione d'onda per la propagazione della perturbazione}\\
    &v_S=\sqrt{\frac{\Gamma_{1,0}P_0}{\rho_0}}&\intertext{adiabatic (Laplacian) sound speed.}
    \end{align*}

    Usando l'equazione di continuit\'a si vede che
    \begin{align*}
    &|\frac{\rho'}{\rho_0}|=\frac{v}{v_S}\\
    &|\frac{\rho'}{\rho_0}|\ll1\ \Rightarrow \ \frac{v}{v_S}\ll1
    \end{align*}
    La teoria lineare \'e valida finch\'e la velocit\'a delle fluttuazioni associata alle onde acustiche \'e minore rispetto alla velocit\'a del suono.
    Per l'equazione per la quantit\'a di moto linearizzata deve essere $\vec{c}\parallel \vec{k}$: la velocit\'a del fluido associato alle onde acustiche adiabatiche \'e parallela alla direzione di propagazione, sono ande di pressione.
    
    \item Forward problem: Matching accuracy of observations with accuracy of theoretical predictions.
    
    \item Metodo asintotico: varie approssimazioni. Accuratezza $10\%$. Le tecniche di inversione numerica hanno accuratezza \numrange{100}{1000} volte superiore ma dipende da un modello solare: dipendenza radiale dei coefficienti nelle equazioni delle oscillazioni.
    
    \item At observed solar frequencies the displacement at surface is approx. radial:
    
    \begin{equation*}
    \frac{\xi_h(R)}{\xi_r(R)}\approx\frac{GM}{R^3}\frac{L}{\omega^2}
    \end{equation*}
    
    \item L'approssimazione di Cowling \'e troppo grossolana se paragonata con l'accuratezza delle osservazioni \num{e-4}.
    (Vedi Robe68 JCD84)
    
    \item Confronto accuratezza asintotica vs numerica vs accuratezza osservazioni
    
    \item Procedure for determine n for computed modes of oscillation: Scufflaire 74, Osaki 75.
    
    \item Cowling approximation: System of equation of second order: Sturm-Liuville problem.
    \item Classification is invariant under continuus variation of equilibrium model: $\lambda=0$ Cowling approximation, $\lambda=1$ full case.
    
    \item Numerical inaccuracy: Van der raay, Palle roca cortes 1986.
    
    \item Mathematical classification often doesn't reflect the physical nature of the modes (Osaki75)
    
    \item Integrated energy: (relative) kinetic energy within a mode
    \begin{equation*}
    E_{n,l}=\frac{\int_0^R[\xi^2_r(r)+l(l+1)\xi_h^2(r)]\rho r^2\,dr}{4\pi M[\xi^2_r(R)+l(l+1)\xi_h^2(R)]}
    \end{equation*}
    
    \item dispersione energia del modo: effetti non adibatici (superficie), effetti non lineari (accoppiamento con altri modi, accoppiamento con flussi di materia)
    
    \item Trapping of modes.
    
    \item Reflection due to increase of $\omega_c$ in external layers before adiabatic approx break down: for what modes ??
    
    
    \item Identificazione dei modi: identificazione dei promontori nel diagramma \dgndi{} and of the lines in the power spectrum of the full disc oscillation signal.
    \item extrapolation to infinite number of grid points (shibahashi osaki 81)
    \item Pulsational unstable: self-excited oscillations.
    \item Eigenfrequencies for an infinite number of grid point is extrapolated $\nu_N=\nu_{\infty}+\frac{a}{N^2}$
    \item Forward problem: asymptotic expression for frequencies.
    \item dipendenza parametrica modello forward problem: asymptotic vs numerical.
\end{itemize}
}

In questa sezione descrivo le caratteristiche dei modi normali del Sole e come la struttura del interna del Sole influisce sulle frequenze.

Quando le frequenza sono molto grandi (per i modi p) o molto piccole (per i modi g) \'e possibile ricavare soluzioni analitiche approssimate delle equazioni delle oscillazioni. 

\section{Perturbazioni lineari adiabatiche.}

\subsection{Perturbatione dello stato di equilibrio.}

\begin{todo}{Cerca ampiezza media oscillazioni superficie ($\exv{v_{osc}}$)}
La piccola ampiezza delle oscillazioni giustifica l'uso solo del termine lineare dell'espansione.

\end{todo}

Descrivo le oscillazioni come piccole perturbazioni attorno allo stato di equilibrio stazionario (gli effetti non lineari sono dell'ordine di $\frac{v}{c_s}$ dove v \'e l'ampiezza dell'oscillazione):

\begin{align*}
&P(\vec{r},t)=P_0(\vec{r})+P'(\vec{r},t)&\intertext{$P'(\vec{r},t)$ \'e la perturbazione euleriana, quindi, detto $\delta\vec{\xi}$ lo spostamento della particella di fluido a causa della perturbazione}\\
&\Lvar{P(\vec{r})}=P(\vec{r}+\Lvar{\vec{\xi}})-P_0(\vec{r})=P'(\vec{r})+\Lvar{\vec{\xi}}\cdot\nabla P_0&\intertext{la velocit\'a dell'elemento di fluido dovuta alla perturbazione \'e}\\
&\vec{v}=\PDof{t}(\Lvar{\vec{\xi}})
\end{align*}

\begin{todo}{Particular solution/Perturbed solution}
Equations of conservation for stellar structure form a system of non-linear, partial differential equations. If an unperturbed solution is known we are often interested in finding another solution ''perturbed'' which differs only slightly from the unperturbed (we may think of the two solutions as representing possible future of the fluid differing from each others because of different initial conditions).

Expressing dependent variable of perturbed solutions as the sum of corr. dependent vars of unperturbed solution, neglecting all powers above the first and product of variations we obtain a system of partial differential equation whose solution gives the behaviour of the variation: the resulting set of equations is linear.

\end{todo}

Ricavo l'equazione del moto perturbato
\begin{align}
&\intertext{sostituisco nell'equazione del moto}
    &\rho\TDof{t}v\indices{_i}=\rho(\PDy{t}{v\indices{_i}}+v\indices{_j}\partial\indices{_j}v\indices{_i})=-\nabla P\indices{_i}+\rho\vec{g}\indices{_i}&\intertext{ le grandezze perturbate e sottraendo l'equazione statica ottengo}\nonumber\\
&\rho_0\PtwoDy{t}{\Lvar{\vec{\xi}}}=\rho_0\PDy{t}{\vec{v}}=-\nabla P'+\rho_0\vec{g}'+\rho'\vec{g}_0\label{eq:emper}\\
&\vec{g}'=-\nabla\Phi',\ \nabla^2\Phi'=4\pi G\rho'\nonumber
\end{align}

Analogamente per l'equazione di continuit\'a ottengo
\begin{equation}
\rho'+\div{(\rho_0\Lvar{\vec{\xi}})}=0\label{eq:contper}
\end{equation}

\subsection{Adiabatic approximation}

I tempi caratteristici per scambio di calore sono maggiori del periodo delle pulsazioni


\begin{align*}
&\TDy{t}{q}=\frac{1}{\rho(\Gamma_3-1)}(\TDy{t}{P}-\frac{\Gamma_1P}{\rho}\TDy{t}{\rho})=\epsilon-\frac{1}{\rho}\scap{\nabla}{F}&\intu{energy equation (rate heat gain/loss)}\\
&\frac{1}{\rho c_P}\nabla\cdot(\frac{4acT^3}{3\kappa\rho}\nabla T)\approx\frac{4acT^4}{3\kappa\rho^2c_PH}=\frac{T}{\tau_R}&\intertext{$\tau_R$ tempo scala radiativo, H lunghezza caratteristica, in cgs:}\\
&\tau_R=\num{e12}\frac{\kappa\rho^2H^2}{T^3}
\end{align*}

Per valori caratteristici solari ($\kappa=1$, $\rho=1$, $T=\num{e6}$, $H=\num{e10}$) ho $\tau_R\approx\SI{e7}{\year}\approx\tkh{}$, per valori caratteristici della zona convettiva ($\kappa=100$, $\rho=\num{e-5}$, $T=\num{e4}$, $H=\num{e9}$) ho $\tau_R\approx\SI{e3}{\year}\approx\tkh{}$.


In the inner part the nuclear term correspond to characteristic time $\tau_{\epsilon}\approx\frac{c_PT}{\epsilon}\approx\tkh{}$.

Confronto $\frac{T}{\tau_R}$, $\frac{T}{\tau_{\epsilon}}$ con $\TDy{t}{T}\approx\frac{T}{\Pi_{osc}}$ con $\Pi_{osc}\approx\si{\minute}-\si{\hour}$: heating term is generally very small compared with time derivative term.

Il moto di una elemento di fluido \'e descritto dalla relazione adiabatica


\begin{align*}
&\TDy{t}{P}=\frac{\Gamma_1P}{\rho}\TDy{t}{\rho}
\end{align*}

Approssimazione adiabatica non pi\'u valida vicino alla superficie solare dove i tempi per lo scambio di calore sono pi\'u brevi.

La condizione di perturbazione adiabatico linearizzata \'e
\begin{align}
&\PDy{t}{\Lvar{P}}-\frac{\Gamma_{1,0}P_0}{\rho_0}\PDy{t}{\Lvar{\rho}}=0\nonumber&\intertext{che integrata rispetto a t ed in funzione della variazione euleriana diventa}\nonumber\\
&P'+\Lvar{\vec{\xi}}\cdot\nabla P_0=\frac{\Gamma_{1,0}P_0}{\rho}(\rho'+\Lvar{\vec{\xi}}\cdot\nabla\rho_0)\label{eq:adper}
\end{align}

\subsection{Separazione variabili spaziali e temporali.}

Dall'equazione del moto \ref{eq:emper} si vede che
\begin{align*}
&\hat{r}\cdot(\rot{\PtwoDy{t}{\vec{\xi}}})=0&\intertext{cio\'e}\\
&\PDof{\theta}(\sin{\theta}\xi_{\phi})-\PDy{\phi}{\xi_{\theta}}=0&\intertext{quindi \'e possibile ricavare la componente tangenziale della perturbazione da una funzione scalare e dato che sono interessato alle oscillazioni }\\
&\vec{\xi}=\exp{i\omega t}(\xi_r(r),\xi_h(r)\PDof{\theta},\frac{\xi_h(r)}{\sin{\theta}}\PDof{\phi})Y_l^m(\theta,\phi)&\intertext{Ho introdotto le funzioni armoniche sferiche che soddisfano:}\\
&L^2Y_l^m=-\frac{1}{\sin{\theta}}\PDof{\theta}(\sin{\theta}\PDy{\theta}{Y_l^m})\\
&+\frac{1}{\sin^2{\theta}}\PtwoDy{\phi}{Y_l^m}=-r^2\nabla_h^2Y_l^m=l(l+1)Y_l^m
\end{align*}

La variazione euleriana di densit\'a, pressione, potenziale gravitazionale sono espressi
\begin{align*}
&(\rho_1,P_1,\Phi_1)=\exp{i\omega t}[\rho_1(r),P_1(r),\Phi_1(r)]Y_l^m
\end{align*}

\subsection{Frequenze di oscillazione discrete.}

Utilizzo l'equzione del moto ~\ref{eq:emper} e l'equazione di continui\'a~\ref{eq:contper} per eliminare $\xi_h(r)$ dall'equazione del moto
\begin{align}
&\frac{1}{r^2}\TDof{r}(r^2\xi_r)-\frac{\xi_rg}{c^2}+\frac{1}{\rho_0}(\frac{1}{c^2}-\frac{l(l+1)}{r^2\omega^2})P_1\nonumber\\
&-\frac{l(l+1)}{r^2\omega^2}\Phi_1=0\nonumber\\
&\frac{1}{\rho_0}(\TDof{r}+\frac{g}{c^2})P_1-(\omega^2-N^2)\xi_r+\TDy{r}{\Phi_1}=0\label{eq:eigenomega}\\
&\frac{1}{r^2}\TDof{r}(r^2\TDy{r}{\Phi_1})-\frac{l(l+1)}{r^2}\Phi_1-\frac{4\pi G\rho_0}{g}N^2\xi_r\nonumber\\
&-\frac{4\pi G}{c^2}P_1=0\nonumber
\end{align}

ho definito $N^2=g(\frac{1}{\Gamma_1P_0}\TDy{r}{P_0}-\frac{1}{\rho_0}\TDy{r}{\rho_0})$ e $S_l^2=\frac{l(l+1)c^2}{r^2}\approx k_h^2c^2$ con

\begin{align*}
&g=-\frac{1}{\rho_0}\TDy{r}{P_0}\\
&c^2=\frac{\Gamma_1P_0}{\rho_0}
\end{align*}


Il sistema di equazione ~\ref{eq:eigenomega} ha soluzione con le opportune equazioni al contorno per un insieme discreto di valori delle frequenze $\omega_{nlm}$, l'ordine angolare non compare nelle equazioni quindi gli autovalori $\omega_{nlm}$ sono $2l+1$ degeneri.

\subsection{Condizioni al contorno}

Abbiamo bisogno di 4 condizioni

\begin{itemize}
\item Due condizioni per $r=0$ punto regolare: le perturbazioni sono non singolari al centro del Sole, $r=0$.

\begin{equation*}
P'=0,\ \Phi'=0
\end{equation*}

Expansion near zero of solutions

\begin{align*}
&(l\neq0):\ \xi_r\propto r\expy{l-1};\ (l=0):\ \xi_r\propto r\\
&P',\ \Phi'\propto r^l
\end{align*}

\item Alla superficie solare richiediamo la continuit\'a di $\Lvar{\nabla\Phi}$ e che non si abbia dispersione verso l'esterno.

Outside the star $\rho'=0$ and Poisson equation can be solved by solution vanishing at infinity $\Phi'=Ar\expy{-l-1}$:
\begin{equation*}
\TDy{r}{\Phi'}+\frac{l+1}{r}\Phi'=0,\ r=\rsun{}    
\end{equation*}

The second condition depends on treatment of stellar atmosphere (Vedi chap 5 of lecture note on stellar oscillations: pg 103, (5.50)). It's reasonable that the boundary is free, no force acts on it: the star can be considered an isolated system. This is equivalent to requiring pressure constant at perturbed surface.

\begin{align*}
&\Lvar{P}=P'+\xi_r\TDy{r}{P}=0
\end{align*}

\end{itemize}

\subsection{Variabili adimensionali.}

Introduco le variabili adimensionali, che caratterizzano la perturbazione

\begin{align*}
&\eta_1=\frac{1}{r}\xi_r\\
&\eta_2=\frac{1}{gr}(\frac{P'}{\rho}+\Phi')\\
&\eta_3=\frac{1}{gr}\Phi'\\
&\eta_4=\frac{1}{g}\PDy{r}{\Phi'}\\
&\eta_i=\eta_i(r)Y_l^m(\theta,\phi)\exp{i\omega t}
\end{align*}

e riscrivo l'equazione del moto

\begin{equation*}
-\frac{\omega^2}{g}\vec{\xi}=[W(\eta_1-\eta_2+\eta_3)+(1-U)\eta_2]\hat{r}-r\nabla\eta_2
\end{equation*}

in funzione delle grandezze $U,V,W$ che caratterizzano lo stato di equilibrio del Sole

\begin{align*}
&U=\frac{r}{m}\PDy{r}{m}=\frac{1}{g}\PDy{r}{(gr)}\\
&V=-\frac{r}{P}\PDy{r}{P}=\frac{g\rho r}{P}\\
&W=\frac{r}{\rho}\PDy{r}{\rho}-\frac{r}{P\gamma_{Ad}}\PDy{r}{P}
\end{align*}

posto $\gamma_{ad}=\Dcvar{\TDly{\rho}{P}}{ad}=\Gamma_1$

La parte tangenziale dell'equazione del moto
\begin{align*}
&\frac{\omega^2}{g}\xi_{\theta}=\PDy{\theta}{\eta_2},\ &\frac{\omega^2}{g}\xi_{\phi}=\frac{1}{\sin{\theta}}\PDy{\phi}{\eta_2}
\end{align*}
sostituita nell'equazione di continuit\'a ($\scap{\nabla}{\xi}$), definita la frequenza adimensionale 
\begin{equation*}
\frac{\omega^2r}{g}=C\sigma^2:\ \sigma^2=\omega^2\frac{R^3}{GM}
\end{equation*}

permette di eliminare la dipendenza dalle variabili angolari
\begin{align*}
&r\PDy{r}{\eta_1(r)}=(3-\frac{V}{\gamma_{Ad}})\eta_1(r)+[\frac{l(l+1)}{C\sigma^2}\\
&+\frac{V}{\gamma_{Ad}}]\eta_2(r)-\frac{V}{\gamma_{Ad}}\eta_3
\end{align*}

mentre la parte radiale dell'equazione del moto

\begin{equation*}
r\PDy{r}{\eta_2(r)}=(W+C\sigma^2)\eta_1(r)+(1-U-W)\eta_2(r)+W\eta_3
\end{equation*}

dalla definizione di $\eta_3$

\begin{equation*}
r\PDy{r}{\eta_3}=(1-U)\eta_3(r)+\eta_4
\end{equation*}

infine l'equazione di Poisson \'e equivalente a
\begin{equation*}
r\PDy{r}{\eta_4}=-UW\eta_1+\frac{UV}{\gamma_{Ad}}\eta_2+[l(l+1)-\frac{UV}{\gamma_{Ad}}]\eta_3-U\eta_4
\end{equation*}



Abbiamo ottenuto quattro equazioni differenziali a coefficienti reali che dipendono dallo stato di equilibrio del modello stellare per le variabili adimensionali $\eta_i(r)$: un problema agli autovalori per $\sigma^2$, si pu\'o vedere che \'e autoaggiunto e quindi le autofunzioni corrispondenti ad autovalori diversi sono ortogonali: gli autovalori sono reali quindi posso avere un comportamento oscillante nel caso di stabilit\'a o esponenziale nel caso instabile.

Il sistema non dipende da m: le soluzioni sono $(2l+1)$ volte degeneri: la degenerazione \'e rimossa dalla rotazione ($\frac{\Omega}{\omega}\approx\num{e-4}$) o effetti gravitazionali di altri corpi.


\section{Stabilit\'a dei modi di oscillazione.}

\begin{todo}{Stabilit\'a oscillazioni nonradiali adiabatiche}

\end{todo}

\section{Propriet\'a generali delle oscillazioni adiabatiche.}

\begin{todo}{Frequenza di \bv{}.}
qui o nella sottosezione precedente??
Kippenhan: 40.3, eigenspectra
Dalsgaard: notes 5.3Pg 83-
\end{todo}

\begin{figure}[!ht]
\centering
\includegraphics[width=\textwidth, height=0.9\textheight,keepaspectratio]{propagationAG}
\caption{Regioni di propagazione.}
\end{figure}


\begin{todo}{dalsgaard 2005}
homology argument: scaling factor $\sqrt{GM}$
\end{todo}


\begin{todo}{Soluzioni numeriche e comportamentpo asintotico}
La soluzione numerica \'e dipendente dal modello di equilibrio: per il modello stellare $M4K$ viene riportata un precisione di $\SI{0.02}{\micro\hertz}$ (accuratezza \num{e-5}), mentre differenti valori della costante G fra quelli usati in letteratura risultano in differenze nelle frequenze calcolate di \numrange{-0.35}{-0.08}\si{\micro\hertz}, maggiori di quelle che risultano da differenti schemi di integrazione numerica.
\end{todo}

\begin{todo}{Numerical technique}
Inter-comparison of the g-, f- and p-modes calculated using different oscillation codes for a given stellar model

\end{todo}

\begin{todo}{Continous variation of parameter}
Problems in cowling vs full
\end{todo}

La soluzione numerica delle equazioni \eqref{eq:eigenomega}



\begin{figure}[!ht]
\centering
\includegraphics[width=\textwidth, height=0.9\textheight,keepaspectratio]{omega-l}
\caption{Modi di oscillazion. plot omega vs l..}
\end{figure}

mostra due differenti comportamenti. Uso l'approssimazione asintotica per determinare la natura delle oscillazioni nelle due zone.

\clearpage

\subsection{Comportamento asintotico}


Per determinare la struttura dello spettro delle oscillazioni introduciamo l'approssimazione di Cowling (\cite{cow41oscillations}) cio\'e trascuriamo la perturbazione del potenziale gravitazionale. Quindi il sistema si riduce al secondo ordine

\begin{align}
&\frac{1}{r^2}\TDof{r}(r^2\xi_r)-\frac{\xi_rg}{c^2}+\frac{1}{\rho_0}(\frac{1}{c^2}-\frac{l(l+1)}{r^2\omega^2})P_1=0\label{eq:cowosc}\\
&\frac{1}{\rho_0}(\TDof{r}+\frac{g}{c^2})P_1-(\omega^2-N^2)\xi_r=0\nonumber
\end{align}

Considero i limiti asintotici di alte e basse frequenze: in entrambi ottengo un problema del tipo di Sturm-Liuville

\begin{todo}{Sturm-Liouville theory}
%https://en.wikipedia.org/wiki/Sturm%E2%80%93Liouville_theory
\end{todo}

\begin{todo}{Modi stabili/instabili}
Cox ??
\end{todo}

\begin{itemize}
\item Per $\omega\to\infty$:

Lo spettro \'e discreto con punto di accumulazione a $\omega=\infty$.
Le oscillazioni sono prodotte da onde acustiche in cui la forza dominante \'e fornita dalla pressione, chiamati modi p, ordinati in base al numero di zeri di $\xi_r$ fra il centro e la superficie. I modi p sono stabili.

\item Per $\omega\to0$:

Lo spettro \'e discreto con punto di accumulazione a $\omega=0$.
Il moto \'e determinato dalla forza di gravit\'a, chiamati modi g (ordinati secondo il numero di nodi radiali). La stabilit\'a dei modi g \'e detrminata dalla stabilit\'a convettiva: dato che $\omega^2_{Ad}=-grW$ il criterio di instabilit\'a convettiva si traduce in $rW>0$. Se $W<0$ in tutta la stella tutti i modi g sono stabili ($g_+$), se esistono zone in cui $W>0$ esistono anche modi g instabili ($g_-$).
\end{itemize}

Lo spettro solare \'e la combinazione dei modi parziali precedenti; il modo f separa  i modi g e p: non ha nodi in direzione radiale.

\subsection{Relazione di dispersione per i modi gravo-acustici.}

Approssimo il comportamento spaziale delle oscillazioni con quello di onda piana
\begin{align*}
&\vec{\xi}\propto\exp{i\scap{k}{x}},\ \vec{k}=k_r\hat{r}+\vec{k}_h\\
&S_l^2=\frac{l(l+1)c^2}{r^2}\approx k_h^2c^2
\end{align*}
e i coefficienti delle equazioni \ref{eq:cowosc} costanti ( approssimazione valida se la lunghezza d'onda delle perturbazioni \'e molto minore della scala caratteristica di variazione dei coefficienti).

\begin{todo}{Per poter parlare di onde}
Per poter parlare di onde devo assumere che la variazione di $P_0$ e $\rho_0$ abbiano lunghezze caratteristiche maggiori delle lunghezze di interesse (short-wave acustic):
\begin{align*}
&\PtwoDy{t}{\rho'}=-v_S^2\nabla^2\rho'\intertext{equazione d'onda per la propagazione della perturbazione}\\
&v_S=\sqrt{\frac{\Gamma_{1,0}P_0}{\rho_0}}&\intertext{adiabatic (Laplacian) sound speed.}
\end{align*}



Usando l'equazione di continuit\'a si vede che
\begin{align*}
&|\frac{\rho'}{\rho_0}|=\frac{v}{v_S}\\
&|\frac{\rho'}{\rho_0}|\ll1\ \Rightarrow \ \frac{v}{v_S}\ll1
\end{align*}
La teoria lineare \'e valida finch\'e la velocit\'a delle fluttuazioni associata alle onde acustiche \'e minore rispetto alla velocit\'a del suono.
Per l'equazione per la quantit\'a di moto linearizzata deve essere $\vec{c}\parallel \vec{k}$: la velocit\'a del fluido associato alle onde acustiche adiabatiche \'e parallela alla direzione di propagazione, pressure force supply the restoring force.

\end{todo}

Manipolando il sistema \ref{eq:cowosc} inserendo perturbazioni della forma (conservazione energia) $\xi_r\propto\rho_0\expy{-\frac{1}{2}}\exp{ik_rr}$, $P_1\propto\rho_0\expy{\frac{1}{2}}\exp{ik_rr}$:

\begin{align*}
&\frac{1}{r^2}\TDof{r}(r^2\xi_r)-\frac{\xi_rg}{c^2}+\frac{1}{\rho_0}(\frac{1}{c^2}-\frac{l(l+1)}{r^2\omega^2})P_1=0\\
&\frac{1}{\rho_0}(\TDof{r}+\frac{g}{c^2})P_1-(\omega^2-N^2)\xi_r=0\\
\\
&\xi_r=\frac{1}{(\omega^2-N^2)}[\frac{1}{\rho_0}(\TDof{r}+\frac{g}{c^2})]P_1\\
&\TDof{r}\xi_r=-\frac{2}{r}\xi_r+\frac{\xi_rg}{c^2}+\frac{1}{\rho_0}(\frac{1}{c^2}+\frac{l(l+1)}{r^2\omega^2})P_1\\
\\
&\TDof{r} \frac{1}{\rho_0}(\TDof{r}+\TDof{r} \frac{g}{c^2})P_1-(\omega^2-N^2)\TDof{r} \xi_r=0
\end{align*}

\begin{todo}{relazione dispersione stix pg 156 (5.33)}

Gough pg 792 eq 25-26

\end{todo}

Se considero $g$, $N$ e $c$ lentamente variabili rispetto alla lunghezza d'onda delle perturbazioni lo stesso vale per la lunghezza caratteristica della densit\'a $H=-\frac{\rho_0}{\TDy{r}{\rho_0}}=(\frac{g}{c^2}+\frac{N^2}{g})\expy{-1}$ e la frequenza di taglio acustica $\omega_A=\frac{c}{2H}$. Scrivo la relazione di dispersione

\begin{align}
&k_r^2=\frac{\omega^2-\omega_A^2}{c^2}+S_l\frac{N^2-\omega^2}{c^2\omega^2}\label{eq:localdispersion}\\
&=\frac{\omega^2}{c^2}(1-\frac{\omega_{l,+}^2}{\omega^2})(1-\frac{\omega_{l,-}^2}{\omega^2})\nonumber
\end{align}

\begin{figure}[!ht]
\centering
\includegraphics[width=\textwidth, height=0.9\textheight,keepaspectratio]{freqcaratt}
\caption{Frequenze caratteristiche.}
\label{fig:freqcaratt}
\end{figure}

\clearpage


\section{Regioni di propagazione.}

\begin{figure}[!ht]
\centering
\includegraphics[width=\textwidth, height=0.8\textheight,keepaspectratio]{khomeagisot}
\caption{Diagramma frequenza numero d'onda orizzontale per atmosfera isoterma.}
\label{fig:khomeagisot}
\end{figure}

\begin{figure}[!ht]
\centering
\includegraphics[width=0.9\textwidth, height=\textheight,keepaspectratio]{pgmodesC}
\caption{Cavit\'a risonanti per modi p e g.}
\label{fig:propagationAG}
\end{figure}

Il comportamento oscillatorio richiede $k_r^2>0$.

I punti di inversione per le onde acustiche sono definiti da 
\begin{align*}
    &\omega^2=\frac{l(l+1)c}{r^2}&\intertext{large $k_hH$}\\
    &\omega=\omega_A&\intertext{small $k_hH$}
\end{align*}

per i modi g da
\begin{align*}
    &\omega=N&\intertext{large $k_hH$.}
    &\omega=(\frac{\omega_A}{N})ck_h&\intertext{small $K_hH$.}
\end{align*}

\'E possibile analizzare tramite metodo  JWKB il sistema di equazioni delle oscillazioni del secondo ordine in approssimazione di Cowling, previa oppurtuna trasformazione, da cui si ottiene la relazione valida per i modi
\begin{equation}\label{eq:jwkb}
\omega\int_{r_1}^{r_2}[1-\frac{\omega_A^2}{\omega^2}-\frac{S_l^2}{\omega^2}(1-\frac{N^2}{\omega^2})]\expy{\frac{1}{2}}\frac{dr}{c}\approx\pi(n-\frac{1}{2})
\end{equation}
dove $r_1$ e $r_2$ sono due zeri consecutivi del numero d'onda radiale e l'integrazione \'e in una regione di propagazione. Nel caso dei modi p e assumendo $S_l\ll\omega$ vicino al punto di inversione superiore ho

\begin{equation}\label{eq:jwkbmodep}
\omega\int_{r_1}^{r_2}[1-\frac{S_l^2}{\omega^2}]\expy{\frac{1}{2}}\frac{dr}{c}\approx\pi(n-\alpha{\omega})
\end{equation}

\clearpage

\subsection{Cavit\'a acustiche.}
Per grandi $\omega$ ~\ref{eq:localdispersion} si riduce alla relazione di dispersione acustica 

\begin{equation*}
\omega^2=c^2(k_r^2+k_h^2)
\end{equation*}


Posso ricavare il raggio di inversione del moto in direzione radiale $k_r=0$ dalla relazione di dispersione per onde onde acustiche, da cui segue
\begin{equation}
\frac{c(r_i)}{r_i}=\frac{\omega}{l(l+1)}
\end{equation}

Maggiore \'e il grado l (piccolo $\lambda_h$) meno profonda \'e la cavit\'a: sono riflesse verso la superfice quando la velocit\'a del suono \'e aumentata fino alla loro velocit\'a di fase orizzontale; la profondit\'a della cavit\'a acustica varia con il variare della scala orizzontale dell'onda. (Top  convection zone down to the level at which refraction due to sound speed increasing $c\propto\sqrt{T}$ turn the wave around when $c=\frac{\omega}{k_h}$)

Stima profondit\'a cavit\'a acustica
\begin{align*}
    &T=\Dcvar{\TDy{z}{T}}{Ad}\delta&\intu{$\delta$ \'e la profondit\'a sotto la fotosfera}\\
    &T=\Dcvar{\TDy{z}{T}}{Ad}=\frac{T}{P}\TDly{P}{T}|_{Ad}\TDy{z}{P}=\frac{\gamma-1}{\gamma R}g=\frac{g}{c_P}\\
    &c^2=(\gamma-1)g\delta&\intertext{da $c=\frac{\omega}{k_h}$ segue:}\\
    &\delta=\frac{\omega^2}{k_h^2(\gamma-1)g}
\end{align*}
minore la lunghezza d'onda orizzontale pi\'u sottile la cavit\'a.

Vicino alla superficie l'efficienza della convezione diminuisce, il gradiente di temperatura diventa fortemente sopra-adiabatico e la fraquenza critica $\omega_A$ aumenta notevolmente: le onde acustiche con periodo attorno ai 5-min diventano evanescenti in poche scale di altezza: l'inizio della zona convettiva \'e uno specchio a larga banda per onde acustiche. 

Duvall82

In un grafico $\frac{\omega}{k_h}$ vs $\frac{\pi(n+\alpha)}{\omega}$ i modi p sono rappresentati da un'unica curva. Se considero la differenza di fase
\begin{align}\label{eq:duvall}
&\Delta\phi=\int_{r_t}^{\rsun{}}k_r\,dr=\int_{r_t}^{\rsun{}}(\frac{1}{c^2}-\frac{l(l+1)}{r^2\omega^2})\expy{\frac{1}{2}}\,dr\\
&=F(\frac{\omega}{L})\\
&\Delta\phi=\pi(n+\alpha)
\end{align}
tra i bordi interno ed esterno della cavit\'a acustica per un modo di oscillazione $\Delta\phi=\pi(n+\alpha)$ la costante $\alpha$ \'e necessaria dato che i bordi non sono rigidi.
L'integrale risulta funzione di $\frac{\omega}{k_h}$. 

\begin{figure}[!ht]
\centering
\includegraphics[width=\textwidth, height=0.9\textheight,keepaspectratio]{Duvall}
\caption{Legge di Duvall.}
\end{figure}

\clearpage

\subsection{Cavit\'a risonanti per modi g.}

Nella parte a basse frequenze dei modi g la relazione \ref{eq:localdispersion} si approssima, per $l\neq0$ con

\begin{equation*}
k_r^2=\frac{S_l^2}{c^2}(\frac{N^2}{\omega^2}-1)
\end{equation*}

La regione dei modi g ha come limite superiore N per grandi l, la linea $\omega=\frac{S_lN}{\omega_A}$.

Per i modi g le regioni di propagazione sono quelle per la frequenza \'e minore di entrambi $N$ e $ck_h$.

La struttura degli strati esterni del sole \'e dominata dalla ionizzazione di H e He con conseguente aumento dell'opacit\'a e quindi del gradiente di temperatura in equilibrio radiativo e il calore specifico: il gradiente di temperatura critico per instabilit\'a convettiva $\frac{g}{c_P}$ diminuisce. In questa regione il gradiente di temperatura \'e debolmente super-adiabatico, $N^2<0$: la zona convettiva costituisce una barriera per le onde di gravit\'a interne.

Le onde di gravit\'a sono presenti nelle regioni in cui il gas \'e neutro o completamente ionizzato ($N^2$ grande) e sono riflesse in regioni dove $N$ \'e piccolo o immaginario: ionizzazione parziale, instabilit\'a convettiva, centro del Sole.

Ho cavit\'a risonanti per modi g:
\begin{itemize}
    \item Core radiativo.
    
    Tra la la parte centrale dove $g\to0$ e il fondo della zona convettiva dove $N^2<0$.
    \item Atmosfera.
    
    $N$ ha un massimo in coincidenza del punto $T_m$ nella cromosfera: modi g confinati tra zona convettiva e cromosfera ($\Pi\approx\numrange{180}{800}\si{\second}$).
\end{itemize}


\section{Analisi asintotica}

\begin{todo}{Analisi asintotica}
Vedi stix 5.3??

Heliosismic inference: observed vs predicted frequencies (Simple models and analytic (asyntotic) formula).
\end{todo}

\subsection{JWKB analysis}

\begin{todo}{cos'\'e l'analisi asintotica}
Vedi articolo gough07
\end{todo}

Nell'analisi tramite JWKB si tiene conto del fatto che le oscillazioni non sono puramente acustiche e le propriet\'a del gas non sono omogenee.

I modi osservati sono di alto ordine radiale o grado angolare: uso l'approssimazione di Cowling (ignoro la perturbazione al potenziale gravitazionale $\Phi'$).

\begin{align*}
&\omega\int_{r_1}^{r_2}\sqrt{1-\frac{\omega_c^2}{\omega^2}-\frac{S_l^2}{\omega^2}(1-\frac{N^2}{\omega^2})}\,\frac{dr}{c}\approx\pi(n-\frac{1}{2})\\
&\omega_c=\frac{c^2}{4H^2}(1-2\TDy{r}{H})\\
&H=-(\TDy{r}{\ln{\rho}})\expy{-1}
\end{align*}

\subsection{Asymptotic properties of p modes}

Posso trascurare N e, eccetto vicino alla superficie, $\omega_c\ll\omega$, mentr vicino alla superficie $S_l\ll\omega$ for small/moderate l.

\begin{align*}
&\omega\int_{r_1}^{r_2}\sqrt{1-\frac{\omega_c^2}{\omega^2}-\frac{S_l^2}{\omega^2}}\,\frac{dr}{c}\approx\pi(n-\frac{1}{2})&\intertext{where $r_1=r_t$, $r_2=R_t$. With help of our assumption we can expand the integral and, introducing the function $\alpha(\omega)$ depending only on frequency and near surface behaviour of $\omega_c$.}
\end{align*}

For low degree modes we use the fact that integrand differs from 1 only close to lower turning point close to center for low order mode ($F(w)\approx\int_0^R\frac{dr}{c}-w\expy{-1}\frac{\pi}{2}$)

\begin{align*}
&\nu_{nl}=\frac{\omega_{nl}}{2\pi}\approx(n+\frac{l}{2}+\frac{1}{4}+\alpha)\Delta\nu\\
&\Delta\nu=[2\int_0^R\frac{dr}{c}]\expy{-1}&\intu{is the inverse of twice travel time center/surface. This equation predict uniform spacing in n of frequency of low degree modes (claverie79)}
\end{align*}

Deviazioni da questa legge hanno potenziale diagnostico per la parte interna, infatti estendendo l'espansione di

\begin{equation*}
F(w)=\int_{r_t}^R\sqrt{1-\frac{c^2}{w^2r^2}}\,\frac{dr}{c}
\end{equation*}

fino al termine dipendente dalla variazione di c:

\begin{align*}
d_{nl}=\nu_{nl}-\nu_{n-1,l+2}\approx-(4l+6)\frac{\Delta\nu}{4\pi^2\nu_{nl}}\int_0^R\TDy{r}{c}\,\frac{dr}{c}&\intertext{sound speed is reduced as $\mu$ increases with H to He conversion as star ages: as a result $d_{nl}$ is reduced providing measure of evolutionary state of stars}
\end{align*}

\subsection{Asymptotic g modes}

In inner domain an expansion in terms of $\frac{\omega^2}{S_l^2}$ is possible, while $\frac{\omega^2}{N^2}$ serves as small expansion parameter in outer domain containing the surface, additional domains have to be considered for zeros of $N^2$

For the Sun we have $N^2(r_v)=0$ where $r_v$ marks lower bound of convection zone, matching the respective expansion we have in first order
\begin{align*}
&T_{n,l}=\frac{2\pi^2(n+\frac{l}{2}-\frac{1}{4})}{\sqrt{l(l+1)}}(\int_0^{r_v}\frac{N}{r}\,dr)\expy{-1}=\frac{n+\frac{l}{2}-\frac{1}{4}}{\sqrt{l(l+1)}}T_0
\end{align*}

g modes have equidistant period spacing.


\section{Excitation and damping. Ampiezza delle autofunzioni meccanismi di eccitazione delle oscillazioni solari.}

\subsection{$\kappa$ mechanism}

Suppose in phase of comperession opacity increases: the compressed layer then absorbs energy out of radiative flux toward stellar surface and thus will be heated in excess than mere adiabatic heating.

The subsequent compression will be stronger than preciding one.

\cite{zhe63variable} demonstrated that this mechanism of overstability drives the pulsation of $\delta$ cephei and related variable stars where is particularly effective in layer of He second ionization.

The crucial parameter measuring the opacity variation is 
\begin{equation*}
\kappa_T=\Dcvar{\PDly{T}{\kappa}}{P}
\end{equation*}

It has a maximum in the layer of partial H ionization: in this layer there is a strong driving but we must include contributions from all layers in order to see if a particular mode is excited or damping.

We must abandon adiabatic assumption and use actual energy equation: 
\begin{equation*}
c_P\rho(\PDy{t}{T}-\nad{}\frac{T}{P}\TDy{t}{P})=-\nabla\cdot\vec{F}
\end{equation*}
from \cite{and75nonadiabatic}: the second term on the left describes adiabatic heating/cooling and $\vec{F}$ is the energy flux.

Il sistema di equazioni che descrive le oscillazioni non e pi\'u autoaggiunto e le frequenze sono complesse.

\'e difficile determinare se un modo sia instabile o meno perch\'e \'e necessario tenere conto dello smorzamento causato dalle perdite radiative nell'atmosfera otticamente sottile e dell'interazione con i moti convettivi non stazionari: difficile da stimare.

Since relative growth rate are small, with $Q=\frac{\Re{\omega}}{\Im{\omega}}\approx10^3$ or larger for solar p modes there is not much certainty about sign of $\Im{\omega}$.

\subsubsection{Argument against excitation of solar p modes by means of $\kappa$ mechanism}

The excited/damped oscillator is represented by
\begin{equation*}
\ddot{\xi}-2\beta\dot{\xi}+\omega^2\xi=0
\end{equation*}

The net effects of all excitation and dumping yield the coefficient $\beta$: if excitation wins over damping $\beta>0$, then there is unlimited growth of this mode (the equation above is homogeneous and linear). It's only be means of non-linear terms neglected above and in oscillation equations that the growth could be held.

Before non-linear terms take effect the amplitud should be sizable unlike small amplitudes observed on the Sun.

\subsection{Stochastic excitation by convection}

L'interazione con la convezione e causa di smorzamento: un blob di gas che si muove avanti e indietro nel suo moto convettivo produce attrito come gli atomi agitati da moto termico e collisioni.

D'altra parte il gas racchiuso tra due pareti riflettenti \'e continuamente colpito/perturbato da blob di gas convettivi: analogo di una campana suona in maniera casuale a una trumphet excited with random spectrum and random phase jump.

Formalmente si descrive l'oscillatore con
\begin{equation*}
\ddot{\xi}-2\beta\dot{\xi}+\omega^2\xi=f(t)
\end{equation*}
dove $f(t)$ \'e una forzante stocastica.

The spectrum and amplitudes of excited modes is determined by forcing function.

Observed frequencies are in the range \SIrange{2}{5}{\milli\hertz}: the upper bound comes about because up to $l\approx2000$ ($2Hk_h\approx1$) the atmospheric acustic cutoff is at about \SI{5}{\milli\hertz} almost indipendent of horizontal wavenumber; at larger frequencies there is no total reflection and no eigenoscillations with discrete spectrum. For lower bound at about \SI{2}{\milli\hertz}: for high l there are no p modes at smaller frequencies, we see the smallest radial order including fundamental; at low l the oscillations of low frequencies have upper reflection boundariy so deep below photosphere that at observable layers the amplitude is undetectable with present technique.


\subsection{Wave propagation in atmosphere}

There is no acustic cutoff for frequencies higher than atmospheric value of $\omega_A$: instead of discrete eigenvalue we expect a continuum of propagating acustic waves clearly seen for frequencies above $\approx\SI{5}{\milli\hertz}$. The phase difference between two levels in atmosphere separated by $\Delta r$ is $\Delta\phi=k_r\Delta r$ and icreases with frequencies because $k_r\approx\frac{\omega}{c_s}$ at this high freq.

There is some phase propagation for frequencies below \SI{5}{\milli\hertz} within spectral band where discrete modes exist: the closer the frequency is to atmospheric $\omega_A$ the less perfect is the reflection of eigenmodes.

Staiger's diagram also indicates presence of IGW in solar atmosphere. The signature is the negative phase difference at low frequencies: using dispersion relation for isothermal atmosphere
\begin{equation*}
\frac{k_h^2(\omega^2-N^2)}{\omega^2(\omega^2-\omega_A^2)}+\frac{k_r^2}{\omega^2-\omega_A^2}=\frac{1}{c^2}
\end{equation*}
which for constant $\omega$ is a quadratic surface in $\vec{k}$ space.

The vector of phase propagation $\vec{k}$ is the radius vector, the group velocity, gradient $\PDy{\vec{k}}{\omega}$ is perpendicular to surface $\omega^2$ const.

In the region of propagating acustic wave the surfaces are oblate ellipsoid of revolution with respect to $k_r$ axis because $\omega^2>\omega_A^2$ (and $\omega^2>N^2$): in this case the vertical components of phase velocity and group velocity have the same sign.

A wave having its excitation deep in the atmosphere will propagate its energy upward and if it's of acustic nature will also propagate phase upward.

By contrast the obove dispersion relation will represent one-shell hyperboloid of revolution for IGW where $\omega^2<N^2$ (and $\omega^2<\omega_A^2$): the r component of phase and group velocity have different sign. An IGW excited from below with upward propagating energy will exhibits downward propagating phase (Vedi steigert intorno a \SI{1}{\milli\hertz}).

(IGW possible only for $N^2>0$ are in stably stratified layers the natural substitutes of convection which depends on $N^2<0$)

\subsection{$\epsilon$ mechanism}

The $\epsilon$ mechanism consist in amplified energy production in the phase of maximum compression (diesel engine): the mechanism would operate in region of max He3 accumulation and lead to growing perturbation because of strong T sensitive of \mblock{^3He(^3He,2p)\alpha} reaction of PPI chain. The g modes, having peak amplitude in deep core would most likely be excited.

Instability of g modes producing finite amplitude perturbation would destroy $^3He$ peak: intermittent manner with timescale approx \SI{e8}{\year}.


\chapter{Problema inverso: correzione al modello dalle oscillazioni.}
\PartialToc

\section{Per punti.}

\begin{itemize}
    
    \item astratto forme di inversione: 3 tipi. Primary inversion: asymptotic technique. Secondary inversion: impose thermal equilibrium, energy transport, equation of state, $\kappa$, $\epsilon$. Tertiary inversion impose that the evolution of the Sun is in accordance with standard model. Inversione fornisce dettagli sulla ''microfisica''.
    
    \item By determining frequency of p,g modes we can probe variation of $c(r)$ and $N$.
    \item Principio variazionale
    \item Tesseral modes ($m\neq0$) yield information about Sun's internal rotation and magnetic field (justified ignoring effects of centrifugal and Lorentz force on radial structure)
    \item Variational principle, model dependent inverion methods. Non ''perfetta corrispondenza'' nel modello solare.
    
    \item Rotation: relation data vs $\Omega(r)$ is linear to high approximation
    \begin{equation*}
    \Delta_i=\int_0^RK_i(r)\Omega(r)\,dr+\epsilon_i
    \end{equation*}
    $\Delta_i$ sono le frequenze osservate e $\epsilon_i$ l'errore sulle frequenze osservate.
    
    \item Dzi90: pressure, density in neutrino production regions: knowledge of these thermodynamic function doesn't suffice tpo determine T.
    
    \item Inversion methods: JCD85; Brodsky, Vorontsov 88; JCD, Gough, Thompson 88;  Vorontsov 88; Kosovichev 88; Gough Kosovichev 88.
    
    \item In determinati casi \'e sufficiente confrontare le differenze di frequenza $\delta_{nl}=\nu_{n,l}-\nu_{n-1,l+2}$ piuttosto che le frequenze assolute.
    
    \item Asymptotic method is not valid in innermost part of the Sun: Shibahashi sekii 88 (shart wavelength asymptotic); shibahashi 89; JCD 89 
    
    \item In the structure case relation between structure and multiplet is highly nonlinear.
    \item Exactly calculated eigenfunction: linearization about SSM.
    \item Averaged  multiplet $\nu_{nl}$ carry info about sherical symmetric component of solar structure: testin solar model, info about properties of matter in solar interior.
    \item $\Gamma_1$, $\rho$, $c$ are constrained by freq. directly.
    \item If equation of state and heavy elements abundances are known $\Gamma_1$ can be expressed in terms of a thermodynamical variable and $Y$: $(\frac{P}{\rho},Y)$, $(\rho,Y)$.
    
    \item Gough84:  Inversion of $F(w)$: $c(t)$ without reference to the model.
    
    \item Simple form of asymptotic analysis
    \begin{align*}
    &\frac{\pi(n+\alpha)}{\omega}\approx F(\frac{\omega}{L})\\
    &F(w)=\int_{r_t}^R\sqrt{1-\frac{c^2}{w^2r^2}}\frac{dr}{c}
    \end{align*}
    sistematic error.
    \item differential asymptotic inversion:
    \begin{align*}
    &S_{nl}\frac{\delta\omega_{nl}}{\omega_{nl}}\approx H_1(\frac{\omega_{nl}}{L})+H_2(\omega_{nl})\\
    &S_{nl}=\int_{r_t}^R(1-\frac{L^2c^2}{r^2\omega_{nl}^2})\expy{-\frac{1}{2}}\frac{dr}{c}-\pi\TDy{\omega}{\alpha}\\
    &H_1(w)=\int_{r_t}^R(1-\frac{c^2}{r^2w^2})\expy{-\frac{1}{2}}\frac{\delta_rc}{c}\frac{dr}{c}\\
    &H_2(w)=\frac{\pi}{\omega}\delta\alpha(\omega)
    \end{align*}
    
    \begin{figure}[!ht]
    \centering
    \includegraphics[width=\textwidth, height=0.9\textheight,keepaspectratio]{freqdiff_13}
    \caption{Frequency difference.}
    \end{figure}

    \clearpage
    
    Drastic change of behaviour for mode penetrating beneath base of convection zone
    
    \item $\nu$ sensitive to details of EOS (Berthomieu, lubow 80)
    \item Confronto densit\'a, velocit\'a del suono del modello vs quelle ottenute dalle frequenza misurate
    \item Frequency changes as linear functional of properties of the models (vedi stix 5.3 etc) for small changes.
    \item Changes to the properties of the model are directly infered from observations.
    \item tecniche di inversion. Classe coefficienti lineare: mola, sola, inversion of acustic data (Thomson 1993). Classe parametri lineare: regularized least square,statistical properties of inference from inversion.
    \item Result of inversion using SOLA fig 14
    \item Inversion technique to deduce interior structure and dynamics given frequencies and their splitting.
    \item Inside of our nearest star as gleaned from the study of its vibration.
    \item Fornisce strumente per scoprire dove il modello solare \'e deficitario (correzioni alla legge dei gas perfetti, Z, etc)
    \item chemical constitution low l p modes (Jimenez88, noel84).
    \item EOS: Electrostatic correct5ion to PG law (Debye-Huckel theory). Effects of electrostatic corrections upon eigenvalue spectrum of low degree p modes + Partial electron degeneracy.
    \item Gli aspetti della struttura solare sono determinati direttamente dai dati.
    \item Determinazione struttura idrostatica.
    \item Inversion of dynamical oscillation: relation between pressure and inertia density $\rho$.
    \item $\omega_{n,l}-\omega_{n-1,l+2}$: chemical inhomogeneities.
    \item Error to be assigned to helioseismological determination of physical quantities Q (characterizing solar structure).
    \item Errors evaluation: Var/CoVar matrix.
\end{itemize}


\section{Tecniche asintotiche.}

Per modi di basso grodo \'e possibile espandere al primo ordine l'integrale nella \ref{eq:duvall} $F(w)\approx\int_0^R\frac{dr}{c}-w\expy{-1}\frac{\pi}{2}$ ed esprimere la legge di Duvall tramite
\begin{equation}\label{eq:claverie}
    \nu_{nl}=\frac{\omega_{nl}}{2\pi}\approx(n+\frac{l}{2}+\frac{1}{4}+\alpha)\Delta\nu
\end{equation}
con $\Delta\nu=[\int_0^R\frac{dr}{c}]\expy{-1}$.
La presenza di picchi uniformemente spaziati di modi a basso grado l \'e stata osservata da Cleverie79.

Estendendo ancora l'espansione di \ref{eq:duvall} si ha una misura della variazione di $c$ nel core della stella
\begin{equation}\label{eq:tassoul}
    d_{nl}=\nu_{nl}-\nu_{n-1,l+2}\approx-(4l+6)\frac{\Delta\nu}{4\pi^2\nu_{nl}}\int_0^R\frac{dc}{dr}\frac{dr}{r}
\end{equation}
La velocit\'a del suono \'e ridotta a causa dell'aumentare di $\mu$ durante la fusione di H in He durante l'evoluzione stellare e quindi $d_{nl}$ \'e ridotto.

\begin{todo}{Analisi legge Duval }
Dalsnotes Pg 155
\end{todo}

\section{Linearizzazione della ''variazione'' attorno ad un modello solare.}

\subsection{Principio variazionale}

Riscrivo l'equazione del moto linearizzata nella forma
\begin{equation}
    \omega^2\Lvar{\vec{r}}=\frac{1}{\rho}\nabla p'-\vec{g}'-\frac{\rho'}{\rho}\vec{g}=\mathcal{F}(\Lvar{\vec{r}})
\end{equation}
da cui risulta che in seguito ad una perturbazione del modello di equilibrio $\Lvar{\mathcal{F}}$ le frequenze delle oscillazioni adiabatiche sono determinate da 

\begin{equation}\label{eq:variational}
    \Lvar{\omega^2}=\frac{\int_V\Lvar{\vec{r}}^*\cdot\mathcal{F}(\Lvar{\vec{r}})\rho\,dV}{\int_V|\Lvar{\vec{r}}|^2\rho\,dV}
\end{equation}
$\Lvar{\vec{r}}$ \'e autovalore per il problema imperturbato.




\section{Rotazione.}

Il Sole \'e un rotatore lento.



We want to find a velocity field which in spherical coordinates has the form
\begin{align*}
&\vec{v_0}=(0,0,r\Omega\sin{\theta})=\vecp{\Omega}{r}\\
&\vec{\Omega(r,\theta)}=(\Omega(r,\theta)\cos{\theta},-\Omega(r,\theta)\sin{\theta},0)&\intertext{il vettore velocit\'a angolare \'e funzione di r e $\theta$}
\end{align*}

Without rotation the inertia term is $\rho_0\TDy{t}{\vec{v}}=\rho_0\PtwoDy{t}{\vec{\xi}}$ where there is no mean motion, in case of rotation $\rho_0(\PDof{t}+\scap{v_0}{\nabla})^2\vec{\xi}$.

We consider additional term as a small perturbation

\begin{align*}
&\PDof{t}=i\omega\\
&\omega=\omega_{\alpha}+\Delta\omega_{\alpha}\\
&Y_{\alpha}=Y_{lm}\\
&\rho_0(\omega_{\alpha}^2+2\omega_{\alpha}\Delta\omega_{\alpha})\vec{\xi}\\
&=\nabla P_1-\frac{\rho_1}{\rho_0}\nabla P_0+\rho_0\nabla\Phi_1+2i\omega_{\alpha}\rho_0(\scap{v_0}{\nabla})\vec{\xi}&\intu{equazione del moto al primo ordine nella perturbazione}
\end{align*}

quindi risulta

\begin{align*}
&\Delta\omega_{\alpha}=\frac{i\int\rho_0\xi_{\alpha}^*(\scap{v_0}{\nabla})\xi_{\alpha}}{\int\rho_0\xi_{\alpha}^*\xi_{\alpha}}\\
&\Delta\omega_{\alpha}=\frac{-m\int\rho_0\Omega\xi_{\alpha}^*\xi_{\alpha}\,dV+i\int\rho_0\xi_{\alpha}^*(\vecp{\Omega}{\xi_{\alpha}})\,dV}{\int\rho_0\xi_{\alpha}^*\xi_{\alpha}}\intu{usando $\vec{v_0}=\vecp{\Omega}{r}$}
\end{align*}

Dobbiamo trovare $\Omega(r,\theta)$ dalla differenza $\Delta\omega_{\alpha}$: the problem is linear in $\Omega$ so the shift $\Delta\Omega_{\alpha}$ is of the same order as $\Omega$.

For evaluation of shift formula we must know eigenfunction $\xi_{\alpha}$ of unperturbed state.

Per rotazione puramente radiale $\Omega(r)$
\begin{align*}
&\Delta\omega_{\alpha}=-m\frac{\int_0^{\rsun{}}\rho_0\Omega\{|\xi_r-\xi_h|^2+[l(l+1)-2]|\xi_h|^2\}r^2\,dr}{\int_0^{\rsun{}}\rho_0\{|\xi_r|^2+l(l+1)|\xi_h|^2\}r^2\,dr}\\
&=\int_0^{\rsun{}}K_{\alpha}(r)\Omega(r)\,dr
\end{align*}

nel caso di rotazione dipendente solo da r $\Delta\omega_{\alpha}$ \'e lineare in m, $2l+1$ frequencies with equidistant spacing.

Any given $\Delta\Omega_{\alpha}$ samples angular velocity in the depth range corresponding to $\xi_{\alpha}$.

Le osservazioni della superficie mostrano una dipendenza dalla co-latitudine 

\begin{equation*}
\frac{\Omega(\theta)}{2\pi}=\SI{451.5}{\nano\hertz}-\SI{65.3}{\nano\hertz}\cos^2{\theta}-\SI{66.7}{\nano\hertz}\cos^4{\theta}
\end{equation*}

risultato di un best fit (discrepanze notevoli e variazioni temporali).

For an investigation of the full function $\Omega(r,\theta)$ the whole multiplet $2l+1$ frequencies must be used: deviation from equidistant spacing within the multiplet is typical of latitudinal shear.


\section{Inversione non asintotica.}

In the structure case the relation between structure and multiplet frequencies is highly nonlinear: we perform linearization on the assumption that a solar model close enough to actual solar structure exists.

\subsection{Correzioni struttra idrostatica}

\'E possibile quindi mettere in relazione la differenze tra le frequenze osservate  con quelle calcolate da un modello, $\delta\omega_{nl}=\Omega_{\odot}-\Omega_{Mod}$ e le differenze nella stratificazione idrostatica

\begin{align}
&\frac{\delta\omega_{nl}}{\omega_{nl}}=\int_0^R[K^{nl}_{c^2,\rho}(r)\frac{\delta_rc^2}{c^2}(r)+K^{nl}_{\rho,c^2}(r)\frac{\delta_r\rho}{\rho}(r)]\,dr\\
&+I_{nl}\expy{-1}F_{Surf}(\omega_{nl})\\
&\frac{\delta_rc^2}{c^2}(r)=\frac{[c_{\odot}^2(r)-c_{mod}^2(r)]}{c^2(r)}\\
&\frac{\delta_r\rho}{\rho}(r)=\frac{[\rho_{\odot}(r)-\rho_{mod}(r)]}{\rho(r)}\label{eq:invstructure}
\end{align}

i kernel $K_Q^j$ dipendono dalle autofunzioni del modello, il termine $I_{nl}\expy{-1}F_{Surf}(\omega_{nl})$, $I_{nl}=\int_V|\Lvar{\vec{r}}|^2\rho\,dV$ \'e una correzione dovuta alle differenti condizioni fisiche che si incontrano vicino alla superficie: per basse frequenze si ha riflessione pi\'u in profondit\'a a $\omega=\omega_c$ e quindi risentono meno degli effetti degli strati superficiali.

The analysis in terms of $\frac{\delta_rc^2}{c^2}(r)$ and $\frac{\delta_r\rho}{\rho}(r)$ capture the difference between Sun and model related hydrostatic structure.

\subsection{Correzioni equazione di stato e composizione.}

Since sound speed depends upon $\Gamma_1$ as $c_s^2=\Gamma_1\frac{P}{\rho}$ we can express $\Gamma_1(P,\rho,Y,Z)$ from thermodynamic properties and composition of the gas.

We obtain equivalent formulation of \ref{eq:invstructure} expressing $\delta_rc^2$ in terms of $\delta_rP$, $\delta_r\rho$, $\delta_rY$ and $\delta_r\Gamma_1$

\begin{align}
&\frac{\delta\omega_{nl}}{\omega_{nl}}=\int_0^RK^{nl}_{u,Y}(r)\frac{\delta_ru}{u}(r)\,dr+\int K^{nl}_{Y,u}(r)\delta_rY\,dr\\
&+\int_0^RK^{nl}_{c^2,\rho}(r)(\frac{\delta\Gamma_1}{\Gamma_1})_{int}\,dr+I_{nl}\expy{-1}F_{Surf}(\omega_{nl})\label{eq:diffthermo}&\intertext{allowance in error $(\delta\Gamma_1)_{int}$, difference between $\Gamma_1$ Sun and $\Gamma_1$ model EOS.}
\end{align}

Fatti:
\begin{itemize}
    \item Inversione di $\Gamma_1$ mostra la necessit\'a di tener conto degli effetti relativistici per gli elettroni (average thermal energy approx \SI{1.35}{\kilo\ev} approx $0.3\%$ of $m_e$).
\end{itemize}


\section{(Numerical) Inversion technique.}

\subsection{Least square inversion.}

Parametrize unknown functions $\frac{\delta_rc^2}{c^2}$, $\frac{\delta_r\rho}{\rho}$, $F_{Surf}$ (Slowly variable polynomials), the parameter being determined through regularized least square fitting (Dziembowski90, Antia Basu 94).

\subsubsection{(Regularized least square methods)}
(JCD90).

\subsection{(OLA)}
(Backus, Gibbert 68,70; Gough 85).

%For rotation inversion $\omega=\omega_{0nl}+m\omega_{1nl}$, $\omega_{1nl}=\int_0^1K_{nl}(x)\Omega(x)\,dx+\epsilon_{nl}$ con $x=\frac{r}{R}$ e $\epsilon_{nl}$ errori in $\omega_{1nl}$
%$\ensemble{c_i(r_0)}$

\subsection{SOLA}
(Pijpers, thompson 92).

Let's determine $\frac{\delta_rc^2}{c^2}$

Expression to be minimized

\begin{align*}
&\int_0^R[\mathcal{K}_{c^2,\rho}(r_0,r)-\mathcal{T}(r_0,r)]^2\,dr\\
&+\beta\int_0^R\mathcal{G}_{\rho,c^2}(r_0,r)\,dr+\mu\sum_i\sigma_ic_i(r_0)c_j(r_0)\\
&\mathcal{K}_{c^2,\rho}(r_0,r)=\sum_ic_i(r_0)K_{c^2,\rho}^i(r)&\intu{averaging kernel}\\
&\mathcal{G}_{\rho,c^2}(r_0,r)=\sum_ic_i(r_0)K_{\rho,c^2}^i(r)&\intu{cross-term kernel which controls the undesidered contrib from $\frac{\delta_r\rho}{\rho}$}
\end{align*}

dove $i=(n,l)$ e $\sigma_i$ \'e l'errore su $\frac{\delta\omega_i}{\omega_i}$.

In general we choose coefficient $c_i(r_0)$ such that $\sum c_i(r_0)\frac{\delta\omega_i}{\omega_i}$ provides a localized average of $\frac{\delta f_1(r)}{f_1(r)}$ around $r=r_0$:

\begin{align*}
&\sum_ic_i(r_0)\frac{\delta\omega_i}{\omega_i}=\int_0^R\sum_ic_i(r_0)K_{1,2}^i(r)\frac{\delta f_1(r)}{f_1(r)}\,dr\\
&+\int_0^R\sum_ic_i(r_0)K_{2,1}^i(r)\frac{\delta f_2(r)}{f_2(r)}\,dr\\
&+\sum_ic_i(r_0)\frac{F_{Surf}(\omega_i)}{\omega_i}
\end{align*}

First term is an average of $\frac{\delta f_1}{f_1}$ weighted by a kernel $\mathcal{K}(r,r_0)=\sum_ic_i(r_0)K_{1,2}^i(r)$.

Second terms is the influence of second function on the solution of the first: weighting function $\mathcal{L}_{21}(r_0,r)=\sum_ic_i(r_0)K_{21}^i(r)$.

The third term is the influence of surface.

The coefficient $c_i(r_0)$ are selected to resmble target function, minimize contamination from $\frac{\delta f_2}{f_2}$ via $\mathcal{L}_{21}$ and minimize effect of noise:

are choosen to minimize
\begin{align*}
&\int(\sum_ic_iK_{12}^i)^2\,dr
&+\beta\int(\sum_ic_iK_{21}^1)^2\,dr\\
&+\mu\sum_{ij}c_ic_jE_{ij}
\end{align*}

$\beta$ is a parameter for contribution of second term.

\section{Helioseismic constrain on solar structure}
We use a SSM as starting model about which hydrostatic equations are linearized: see variational principle connecting differences between solar and the model function describing radial structure to corresponding differences in modes frequencies.

From inversion we infer the value of observables 

\begin{align*}
&Q_{\odot}=Q_{Mod}+q(\omega)&\intertext{for a given inversion procedure $\Delta\Omega$ propagate to the helioseismic value of observable $Q_{\odot}$, also we a residual dependence on starting model and regularization procedure}
\end{align*}


Asymptotic approximation for radial eigenfunction (integral equation connectin sound speed $c(r)$ to $\Omega_{nl}$) is inadequate (especially in deep interior)

\subsection{Helioseismological ''correction'' procedure}

\begin{itemize}
\item $\{\Omega\}$ of p-modes $\xrightarrow{\text{inversion}}Q$.
\item Solar Model $\to Q_{Mod}\to \{\Omega_{Mod}\}$.
\item $\Omega_{Mod}$ vs $\Omega_{\odot}\pm\Delta\Omega_{\odot}$: searching for correction q to solar model in order to match $\{\Omega_{Mod}+\omega(q)\}\leftrightarrow\Omega_{\odot}$. ($\omega(q)$: linear perturbation theory)
\end{itemize}

Assumptions:
\begin{align*}
&q=Q_{\odot}-Q_{Mod}\\
&\gamma=\Gamma_{\odot}-\Gamma_{Mod}&\intertext{$P, \rho$ and combination of their derivatives: connected through linearized mechanical equilibrium condition}
\end{align*}

\begin{itemize}
\item Slow variation of unknown functions
    \item Using a thermodynamical relation for $\Gamma(P,\rho,Y)$ one can eliminate one of the functions $q(x),\gamma(x), F(\Omega)$ and assuming $Y=Y_{ph}$ in convection zone and $\gamma=0$ in radiative interior
\end{itemize}


with these additional constraints the unknown function $\gamma$ is related with unknown number $Y_{ph}^{\odot}$, and chosing $U=\frac{P}{\rho}$ we write \autoref{eq:diffthermo}, where
\begin{itemize}
    \item $\delta_ru=u_{\odot}-U_{Mod}=u$.
    \item $y_{ph}=Y_{ph}^{\odot}-Y_{ph}^{mod}$.
\end{itemize}

\subsection{Outer convective zone.}

The quantities characterizing the outer part are $R_b$, $Y_{ph}$ and $c_b$, $\rho_b$ at bottom of convection zone.

\begin{itemize}
    \item He abundance. Helioseismical detemination $Y_{ph}=\numrange{0.226}{0.260}$.
    
    \item Bottom of convection zone.
    
    Transition of temperature gradient between subadiabatic and adiabatic at base of solar convection zone gives rise to clear signature in sound speed: helioseismical measurement of sound speed permits determination of base of convection zone
    
    \begin{align*}
    &\frac{R_b}{\rsun{}}=\numrange{0.710}{0.716}\\
    &c_b=\SIrange{0.221}{0.225}{\mega\meter\per\second}
    \end{align*}
    
    Lower part of convective zone is very close to being adiabatically stratified, $\Gamma\approx\frac{5}{3}$: $P\propto\rho\expy{\frac{5}{3}}$.
    
    $\rho_b$ is an indipendent quantity ($\rho(x)$ in convective zone is determined up to a scaling factor): the helioseismological determination of $\rho_b$ fixes such a factor.
    
    \begin{equation*}
        \rho_b=\SI{0.192}{\gram\per\cubic\cm}
    \end{equation*}
    
\end{itemize}

\subsection{Intermediate part ($0.2<x<0.65$)}

Isothermal sound speed $U=\frac{P}{\rho}$: recostruction of sound speed profile $c^2=\Gamma U$.

Below convective zone $\Gamma=\frac{5}{3}$ with an accuracy of \num{e-3} or better. Helioseis. determination is very accurate in this region $\frac{\Delta U}{U}\leq5 \perthousand$.

\subsection{Inner part ($x<0.2$)}

\appendix
\part{Appendice}

\stopcontents[chapters]


\clearpage
\addcontentsline{toc}{section}{Index}
\printindex

\end{document}