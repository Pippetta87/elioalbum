\documentclass[../main.tex]{subfiles}

\begin{document}

\chapter{Riassunto}
\PartialToc

\section{Riassunto A}
provo tastiera \'
Storicamente le osservazioni delle stelle pulsanti, di cui $\delta$ Cephei \'e un modello esemplare, hanno preceduto l'identificazione di moti periodici sulla superficie del sole, ci\'o \'e dovuto, con riferimento alle propriet\'a intrinseche delle oscillazioni, alla loro ampiezza, alla loro coerenza spaziale e al loro differente periodo; infatti le Cepheidi compiono oscillazioni radiali relative di decimi del raggio stellare e in velocit\'a di decine di chilometri al secondo, mentre la superficie solare oscilla con velocit\'a di $0.5 Km/s$, i singoli modi causano moti nella fotosfera con velocit\'a dell'ordine di $cm/s$, e la variazione del raggio \'e dell'ordine di $\sci{-6}\rsun$. Un comportamento periodico \'e osservato in diverse regioni del diagramma di \hr{} e differenti possono essere i meccanismi di eccitazione e la natura delle perturbazioni. In questo contesto considero le grandezze fondamentali che differenziano le varie fasi dell'evoluzione stellare, le classi di stelle, varie regioni di una stessa stella, o che evidenziano i fenomeni fisici dominanti.

L'osservazione dei moti periodici della fotosfera solare (oscillazione dei 5 minuti,Leigthon 1962) e la scoperta che tali moti periodici sono la sovrapposizione di modi discreti (Deubner 1975) sono le basi osservative dell'eliosismologia. In questo elaborato discuto sommariamente le osservazioni relative  alle oscillazioni con periodo 5 minuti e la loro struttura modale: le osservazioni di Duvall, che mostrano la relazione tra le propriet\'a superficiali delle oscillazioni e il loro comportamento all'interno del sole, e Claverie, che confermano la struttura con picchi regolarmente spaziati, dello spettro delle oscillazioni dei 5 minuti sono particolarmente importanti. Inoltre delineo le tecniche osservative, le problematiche legate alla precisione richiesta dalle osservazioni eliosismologiche e le tecniche di base usate per l'elaborazione dei dati.

Lo scopo principale di questa tesina \'e invece una descrizione teorica dei modi di oscillazione del sole, risolvendo con tecniche approssimate le equazioni che governano l'evoluzione di perturbazioni infinitesime dell'equilibrio idrodinamico; quindi quali propriet\'a della struttura solare li determinino direttamente e come si possano confrontare, attraverso alcune tecniche di inversione di base, i parametri caratteristici di un modello solare con i dati eliosismologici.

Introduco i modi normali per il moto ondoso: suppongo che le grandezze fisiche che determinano il problema dipendano solo dalla distanza dal centro, \'e quindi naturale descrivere l'ampiezza delle oscillazioni  in termini di armoniche sferiche per la dipendenza angolare che sono identificate dalla distribuzione caratteristica delle fasi di oscillazione sulla superficie solare, queste definiscono il grado l del modo normale; le autofunzioni dell'ampiezza dell'oscillazione radiale sono caratterizzate dall'indice n, il cui modulo riflette il numero di zeri dell'ampiezza radiale.


La piccola ampiezza delle oscillazioni mi permette di usare la teoria delle perturbazioni applicata alle equazioni di un un corpo autogravitante in equilibrio idrostatico per ricavare attraverso l'equazione del moto (equazione di Eulero) un'equazione vettoriale agli autovalori per le frequenze di pulsazione. 

Scrivo la relazione di dispersione nella forma pi\'u generale e ricavo il sistema di equazione differenziali che descrive l'evoluzione delle perturbazioni: distinguo una zona in cui si propagano i modi acustici o modi p, con frequenza maggiore della frequenza di Lamb, ed una zona, la zona interna, in cui si propagano i modi g o modi di gravit\'a aventi frequenza minore della frequenza di Brunt-\vai{}, e modi f confinati in superficie. 

Rimane cos\'i definito un operatore lineare che nel caso di oscillazioni adiabatiche consente di determinare le frequenze tramite un principio variazionale utile per problematiche di inversione. Questo tipo di problema \'e comune in fisica teorica quindi esistono molte tecniche numeriche per determinare le frequenze con l'accuratezza necessaria per il confronto con i dati sperimentali ma non tratter\'o questa problematica.

Uso le leggi dell'acustica geometrica per stimare lo spessore in cui sono confinati i modi acustici: dalla relazione dispersione per onde acustiche deduco la distanza dal centro del sole tale che il moto ondoso sia puramente orizzontale, in quel punto ho che la frequenza delle oscillazioni \'e uguale alla frequenza di Lamb. I modi normali di oscillazione della superficie sono prodotti dall'interferenza di un gran numero di onde aventi in comune la distanza dal centro di inversione del moto. Da calcoli accurati risulta che i modi p sono confinati nella parte esterna della zona convettiva. 

I risultati eliosismologici dimostrano la validit\'a dei modelli solari standard, in particolare tutti i modelli introdotti per spiegare la discrepanza tra il flusso di neutrini osservato e previsto non trovano conferma. Di particolare importanza \'e la  misura  dell'abbondanza di elio nel sole, valore che nei modelli solari standard viene variato, insieme al parametro che regola l'efficienza del trasporto energetico nella zona convettiva, per ottenere, determinando numericamente l'evoluzione del modello iniziale, i giusti valori di luminosit\'a e raggio attuali.

Infine mostro come l'inversione eliosismologica fornisca una guida per individuare le zone in cui il modello solare non \'e corretto.

 
\printbibliography[heading=subbibintoc]



\section{Riassunto B}


Il sole \'e una massa di gas autogravitante che supporta numerosissimi modi di oscillazione attorno alla sua posizione di equilibrio. Il loro studio fornisce uno strumento per determinare caratteristiche essenziali della struttura solare. In seguito alla comprensione della struttura spettrale delle oscillazioni solari sono state identificate altre stelle che mostrano oscillazioni con analoga struttura.

Analizzer\'o inizialmente le grandezze astrofisiche fondamentali, le importanti relazioni tra di esse, e le grandezze fondamentali che evidenziano i fenomeni fisici dominanti; passer\'o poi alla descrizione dei dati osservativi riguardo alle oscillazioni solari.

La rivelazione dei moti periodici della fotosfera solare (oscillazione dei 5 minuti: \cite{lei62velocity}) e la scoperta, in misure in cui la superficie solare \'e risolta spazialmente (\cite{deu75observations}) e in misure integrate sull'intero disco solare (\cite{cla79solar}), che tali moti periodici sono la sovrapposizione di modi discreti, sono le basi osservative dell'eliosismologia. 
In questo elaborato discuto molto brevemente le osservazioni relative  alle oscillazioni con periodo 5 minuti e la loro struttura modale e le osservazioni di Duvall (\cite{duv82dispersion}), che mostrano la relazione tra le propriet\'a superficiali delle oscillazioni e il loro comportamento all'interno del sole.

Claverie , che confermano la struttura con picchi regolarmente spaziati, dello spettro delle oscillazioni dei 5 minuti sono particolarmente importanti.

Accenner\'o brevemente alle tecniche osservative ed alle problematiche legate alla precisione richiesta dalle osservazioni eliosismologiche.

Lo scopo principale di questa tesina \'e invece una descrizione teorica dei modi di oscillazione del sole, risolvendo con tecniche approssimate le equazioni che governano l'evoluzione di perturbazioni infinitesime dello stato di equilibrio; quindi quali propriet\'a della struttura solare li determinino direttamente e come si possano confrontare, attraverso alcune tecniche di inversione di base, i parametri caratteristici di un modello solare con i dati eliosismologici.

Introduco quindi i modi normali per il moto ondoso: suppongo che le grandezze fisiche che determinano il problema dipendano solo dalla distanza dal centro, \'e quindi naturale descrivere l'ampiezza delle oscillazioni  in termini di armoniche sferiche per la dipendenza angolare che sono identificate dalla distribuzione caratteristica delle fasi di oscillazione sulla superficie solare, queste definiscono il grado l del modo normale; le autofunzioni dell'ampiezza dell'oscillazione radiale sono caratterizzate dall'indice n, il cui modulo riflette il numero di zeri dell'ampiezza radiale.

La piccola ampiezza delle oscillazioni mi permette di usare la teoria delle perturbazioni lineari applicata alle equazioni di un un corpo autogravitante in equilibrio idrostatico per ricavare attraverso l'equazione del moto (equazione di Eulero) un'equazione vettoriale agli autovalori per le frequenze di pulsazione. Questo tipo di problema \'e comune in fisica teorica quindi esistono molte tecniche numeriche per determinare le frequenze con l'accuratezza necessaria per il confronto con i dati sperimentali ma non tratter\'o questa problematica.


Ricavo il sistema di equazione differenziali che descrive le perturbazioni adiabatiche e scrivo la relazione di dispersione nella forma pi\'u generale: distinguo una zona in cui si propagano i modi acustici o modi p, con frequenza maggiore della frequenza di Lamb, ed una zona, la zona interna, in cui si propagano i modi g o modi di gravit\'a aventi frequenza minore della frequenza di Brunt-\vai{}, e modi f confinati in superficie. 


Rimane cos\'i definito un operatore lineare che nel caso di oscillazioni adiabatiche consente di determinare le frequenze tramite un principio variazionale utile per problematiche di inversione. 

Uso le leggi dell'acustica geometrica per stimare lo spessore in cui sono confinati i modi acustici: dalla relazione dispersione per onde acustiche deduco la distanza dal centro del sole tale che il moto ondoso sia puramente orizzontale, in quel punto ho che la frequenza delle oscillazioni \'e uguale alla frequenza di Lamb. I modi normali di oscillazione della superficie sono prodotti dall'interferenza di un gran numero di onde aventi in comune la distanza dal centro di inversione del moto. Da calcoli accurati risulta che i modi p sono confinati nella parte esterna della zona convettiva. 

I risultati eliosismologici dimostrano la validit\'a dei modelli solari standard. Di particolare importanza \'e la  misura  dell'abbondanza di elio nel sole, valore che nei modelli solari standard viene variato, insieme al parametro che regola l'efficienza del trasporto energetico nella zona convettiva, per ottenere, determinando numericamente l'evoluzione del modello iniziale, i giusti valori di luminosit\'a e raggio attuali.

Infine mostro come l'inversione eliosismologica fornisca una guida per individuare le zone in cui il modello solare non \'e corretto.



In questo elaborato descrivo le basi teoriche e osservative dell'eliosismologia. Come la geofisica studia la propagazione di onde acustiche nell'interno terrestre per studiarne la struttura, cos\'i la sismologia solare osserva le variazioni periodiche dell'atmosfera solare e descrive la dipendenza delle frequenze di pulsazione dalle caratteristiche interne del sole, in particolare densit\'a, velocit\'a del suono, coefficiente adiabatico e accelerazione gravitazionale. 



Precise misure del campo di velocit\'a dell'atmosfera solare evidenziano scale spaziali e temporali privilegiate: sono osservati picchi di energia tra i 3 e i 160 minuti. Mostro che le oscillazioni dei 5 minuti della fotosfera sono causate da l'interferenza di modi di onde acustiche confinati negli strati esterni del sole con profondit\'a diverse.

\printbibliography[heading=subbibintoc]


\chapter{Per punti.}
\PartialToc


\section{Intro grandezze solari e microscopiche caretteristiche.}

 Cosa \'e un'equazione di stato? Cosa \'e un plasma? Cosa caratterizza l'equilibrio solare? 

Il sole \'e una massa di gas autogravitante che supporta numerosissimi modi di oscillazione attorno alla sua posizione di equilibrio. Il loro studio fornisce uno strumento per determinare caratteristiche essenziali della struttura solare. In seguito alla comprensione della struttura spettrale delle oscillazioni solari sono state identificate altre stelle che mostrano oscillazioni con analoga struttura.

Analizzer\'o inizialmente le grandezze astrofisiche fondamentali, le importanti relazioni tra di esse, e le grandezze fondamentali che evidenziano i fenomeni fisici dominanti

\begin{itemize*}
\item Age of the sun (solar system, Star formation (\cite{han12stellar}))
\item Mass (\cite{ber03solar})
\item Equazioni di base della struttura stellare
\item Equazion of motion for spherical symmetry: $\tau_{ff}$, $\tau_{expl}$ ($\S 2.4$ kipp): la stella occupa stati di quasi equilibrio per gran parte della vita $\tau_{nucl}$
\item Kelvin-Helmholtz scale time ($\S 3.1-3.3$ kippen):
\item Equazioni par 4 cox (nella parte equilibrio struttura autogravitante): leggi di conservazione
\item Equazioni struttura solare. Simmetria sferica: cosa trascuro.
\begin{align*}
&\TDy{r}{p}=-\frac{Gm\rho}{r^2}&\intu{Momentum conservation along with Poisson's equation:}\\
&\TDy{r}{m}=4\pi r^2\rho\\
&\TDy{r}{T}=\nabla\frac{T}{p}\TDy{r}{p}\\
&\TDy{r}{L}=4\pi r^2[\rho\epsilon-\rho\TDof{t}\frac{u}{\rho}+\frac{p}{\rho}\TDy{t}{\rho}]
\end{align*}

\item The assumption $\ten{P}=IP$ where P is the thermodynamic pressure imply neglegible molecular and radiative viscosity, large-scale magnetic field and turbolence.

La forza totale agente su un volume V di superficie S \'e
\begin{equation*}
\int_V\vec{f}\rho\,dV+\int_S\vec{t}(\hat{n},\vec{x},t)\,dS
\end{equation*}
dove $\hat{n}$ \'e la normale in ciascun punto di S, e $\vec{f}$ \'e una forza per unit\'a di massa. Definisco il tensore degli sforzi $P\indices{_i_j}$ le cui componenti sono $t_j(\hat{n},\vec{x},t)$ al variare di $\hat{n}=e_i$ 
\begin{equation*}
P_{ij}=t_j(\hat{e}_i,\vec{x},t)% $\hat{n}=\hat{e}_i$
\end{equation*}

La conservazione della quantit\'a di moto richiede
\begin{align}
&\rho\TDy{t}{v\indices{_i}}=-\partial\indices{_j}P\indices{_i_j}+\rho f\indices{_i}\nonumber&\intertext{dove $P_{ji}$ rappresentano le componenti del tensore degli sforzi. Per pressione termodinamica cio\'e trascurando viscosit\'a molecolare e radiativa, campi magnetici e turbolenze}\nonumber\\
&\rho\TDy{t}{\vec{v}}=-\nabla P+\rho\vec{f}\nonumber&\intertext{ottengo quindi la condizione di equilibrio idrostatico $\ddvec{r}=0$:
}\nonumber\\
&\nabla P=\rho \vec{f}\label{eq:idrosta}
\end{align}


\item \sout{temposcala dinamico:}
\item \sout{teorema del viriale.}
\item \sout{viriale (\cite{han12stellar})}
\item \sout{Period-mean density relation}
\item ''Metodi approsimativo per stima periodo pulsazione modo radiale fondamentale. Si applica anche a modi g e p non radiali di basso ordine. Long-wave acustic: wavelength circa dimensione sistema''

\item particle diffusive effect (shu pg 34)
\item Diffusion coefficient non fa differenza tra diffusion e settling
\item variazione composizione chimica: fusione nucleare, settling e diffusione; tempo di mixing per zone convettive (5.5.3 pg 70)
\item Sole overview struttura macroscopica attuale
\item Instabilit\'a convettiva: caratteristiche.
\item convective mixing (chap 5)
\item onde propagazione frequenze plasma lunghezze caratteristiche frequenze di taglio (asymptotic description)
\item Plasma ideale: costante di accoppiamento
\item equazione di stato (stix pg 29) ???
\item The applicability of fluid approach (sh8u gas dynamics)
\item mean free path and plasma frequency (sh8u gas dynamics)
\end{itemize*}

\printbibliography[heading=subbibintoc]


\section{dynamics of solar photosphere e tecniche osservative.}

cosa misuriamo?

Inquadro la situazione osservativa e le grandezze caratteristiche rilevanti.

passer\'o poi alla descrizione dei dati osservativi riguardo alle oscillazioni solari.

L'osservazione dei moti periodici della fotosfera solare (oscillazione dei 5 minuti,Leigthon 1962) e la scoperta che tali moti periodici sono la sovrapposizione di modi discreti (Deubner 1975) sono le basi osservative dell'eliosismologia. Claverie 1979
Inoltre delineo le tecniche osservative, le problematiche legate alla precisione richiesta dalle osservazioni eliosismologiche e le tecniche di base usate per l'elaborazione dei dati.
 
\begin{itemize*}
\item quali moti della fotosfera considero?
\item descrizione strati esterni del sole
\item vari tipologie di moti nella photosfera
\item regular motion (vedi sun as a star)
\item atmosfera solare: profondit\'a ottica, etc
\item Approx di base per equazione di stato atmosfera stellare (\cite{ste74waves})
\item resolved unresolved motion shrot description
\item Fonti di rumore
\item leighton 62 \cite{lei62velocity}
\item ridges in power spectrum: p-modes, essentially acustic modes; differenza fase in funzione della frequenza
\item FFT
\item deubhner 75 \cite{deu75observations}
\item rapporto osservazione strumenti di misura: precisione risp alle osservazioni solari
\item tool for solar observations: tipo di misure, risoluzione spaziale temporale frequenza, lunghezza d'onda velocit\'a, dove sono implementati (vedi cunha et altri)

\item \sout{fase propagation}
\end{itemize*}

\printbibliography[heading=subbibintoc]


\section{onde in gas e perturbazioni lineare (variazioni??) prime approssiamzioni. Equations for stellar oscillations (non radial). Dipendenza angolare e temporale. Onda piana, numero d'onda radiale e tangenziale. Modi acustici.}

La piccola ampiezza delle oscillazioni mi permette di usare la teoria delle perturbazioni lineare applicata alle equazioni di un un corpo autogravitante in equilibrio idrostatico. variazioni e perturbazioni delle equazioni della struttura stellare

\begin{itemize*}
\item non-radial oscillations  (\cite{han12stellar})
\item Giusto chiarezza variabili euleriani lagrnagiane $\S 1.1$ kipp

\item Moto ondoso in gas: acustic gravity waves
\item standing waves
\item discussion of phase relation/propagation
\item Equazioni idrodinamica: variazioni?
\item perturbazioni lineari e approx adiabat. Equazione del moto.
\item Laplacian sound speed. (Cox chap 5)
\item Perch\'e posso fare l'approssimazione di oscillazioni adiabatiche?: Tempi scala scambio di calore.
\item Propagazione onde \cite{tol63waves}
\item natura confinamento onde AG in cavot\'a risonanti: relazuione di dispersione

\item asimptotic behavior, JWKB, interferentza costruttiva, relazione di dispersione, frequenze caratteristiche, modi p, g,f
\item cavit\'a risonante, modi acustici
\end{itemize*}


\subsection{Modi normali equazione di dispersione legge di duval}
Introduco i modi normali per il moto ondoso: suppongo che le grandezze fisiche che determinano il problema dipendano solo dalla distanza dal centro, \'e quindi naturale descrivere l'ampiezza delle oscillazioni  in termini di armoniche sferiche per la dipendenza angolare che sono identificate dalla distribuzione caratteristica delle fasi di oscillazione sulla superficie solare, queste definiscono il grado l del modo normale; le autofunzioni dell'ampiezza dell'oscillazione radiale sono caratterizzate dall'indice n, il cui modulo riflette il numero di zeri dell'ampiezza radiale.

Scrivo la relazione di dispersione nella forma pi\'u generale e ricavo il sistema di equazione differenziali che descrive l'evoluzione delle perturbazioni: distinguo una zona in cui si propagano i modi acustici o modi p, con frequenza maggiore della frequenza di Lamb, ed una zona, la zona interna, in cui si propagano i modi g o modi di gravit\'a aventi frequenza minore della frequenza di Brunt-\vai{}, e modi f confinati in superficie. 

Uso le leggi dell'acustica geometrica per stimare lo spessore in cui sono confinati i modi acustici: dalla relazione dispersione per onde acustiche deduco la distanza dal centro del sole tale che il moto ondoso sia puramente orizzontale, in quel punto ho che la frequenza delle oscillazioni \'e uguale alla frequenza di Lamb. I modi normali di oscillazione della superficie sono prodotti dall'interferenza di un gran numero di onde aventi in comune la distanza dal centro di inversione del moto. Da calcoli accurati risulta che i modi p sono confinati nella parte esterna della zona convettiva. 

Le osservazioni di Duvall, che mostrano la relazione tra le propriet\'a superficiali delle oscillazioni e il loro comportamento all'interno del sole, e Claverie, che confermano la struttura con picchi regolarmente spaziati, dello spettro delle oscillazioni dei 5 minuti sono particolarmente importanti.


\begin{itemize*}
\item simmetria sferica
\item Derivazione sistema ampiezza oscillazione radiale

\item Inversione assoluta
\item (which aspect of stellar structure are accessible to study?).
\item Osservazioni duvall, claveir \cite{duv82dispersion}
\item best fit for n: degree of freedom.
\end{itemize*}

\printbibliography[heading=subbibintoc]

\section{Modello solare e tecniche di inversione}


I risultati eliosismologici dimostrano la validit\'a dei modelli solari standard, in particolare tutti i modelli introdotti per spiegare la discrepanza tra il flusso di neutrini osservato e previsto non trovano conferma. Di particolare importanza \'e la  misura  dell'abbondanza di elio nel sole, valore che nei modelli solari standard viene variato, insieme al parametro che regola l'efficienza del trasporto energetico nella zona convettiva, per ottenere, determinando numericamente l'evoluzione del modello iniziale, i giusti valori di luminosit\'a e raggio attuali.

Infine mostro come l'inversione eliosismologica fornisca una guida per individuare le zone in cui il modello solare non \'e corretto.

\subsection{Modello Solare}

\begin{itemize}

\item Modello solare, parametri del modello, equazioni di base e vincoli osservativi: diagramma di HR e misure spettrometriche della fotosphera, (Convective motion on supergranular scale-local phenomena in atmosphere-global oscillation of the sun), atmosfera, etc. (Sun: chap. 2-4, 6.)


\item  Modello solare. Variational principle, model dependent inverion methods. Non ''perfetta corrispondenza'' nel modello solare.

\item modello solare standard.

\item Rotazione: introduco i kernel
\item Approx di base per equazione di stato interno stellare (\sch{}, kippenhahn, clayton)
\item Equation of state (\cite{han12stellar})
\item Modello solare stellar modelling ( (\cite{han12stellar}))
\item sole \'e stella in sequemza principale
\item ZAMS (\cite{han12stellar})
\item MLT( (\cite{han12stellar}))
\end{itemize}

\subsection{Tecniche di inversione}

\begin{itemize}
\item astratto forme di inversione: 3 tipi

\item Principio variazionale
\item Confronto densit\'a, velocit\'a del suono del modello vs quelle ottenute dalle frequenza misurate
\item tecniche di inversion. Classe coefficienti lineare: mola, sola, inversion of acustic data (Thomson 1993). Classe parametri lineare: regularized least square,statistical properties of inference from inversion. 
\end{itemize}

\printbibliography[heading=subbibintoc]

\stopcontents[chapters]

\end{document}
